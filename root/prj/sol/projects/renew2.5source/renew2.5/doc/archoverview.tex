\chapter{Package Overview}\label{chap:overview}

This chapter introduces the main packages and their use.
In Fig.~\ref{fig:packages} you can see the dependencies
among the various packages. Here, as always in the packages
diagrams in this manual, dependencies that can be inferred
via a sequence of other dependencies are usually not displayed,
even if a direct dependency exists, too.

\singlediagram{packages}{Overview of all Renew packages}


\section{Hierarchy \texttt{CH.ifa.draw}}

Although the class hierarchy
presented here still reflects the general
structure of the original JHotDraw \cite{Gamma98}, it has been heavily
modified. Especially, a parent/child mechanism was added
to associate figures with each other. Moreover, a single graphical
editor may now be responsible for multiple drawing windows.

\begin{description}
\item[\texttt{CH.ifa.draw.framework}] 
  This package contains the main abstractions of JHotDraw 
  in the form of interfaces and some auxiliary classes.
\item[\texttt{CH.ifa.draw.standard}] 
  This package contains some basic implementations of the
  abstractions found in \texttt{CH.ifa.draw.framework}.
  Typically, these implementations require some additional 
  code to be fully functional, but they ensure that tedious
  bookkeeping tasks are taken care of and that some convenience
  methods are available.
\item[\texttt{CH.ifa.draw.figures}]
  This package contains classes for figures of various shapes, but also
  the tools and handles required to create and modify these
  figures.
\item[\texttt{CH.ifa.draw.contrib}] 
  This package contains code for a number of figures that
  were not originally part of JHotDraw.
\item[\texttt{CH.ifa.draw.util}] 
  This package contains miscellaneous classes. Most of these are
  concerned with building the GUI of JHotDraw and doing I/O.
  The class \texttt{Storable} is especially notable, because it
  supports the external storage format for saving drawings.
\item[\texttt{CH.ifa.draw.application}]
  This package contains a complete graphical editor as
  a stand\-alone application. The main class is \texttt{DrawApplication}.
\end{description}

In many classes of this hierarchy, events are used for
propagating state changes. Make sure to understand, at least vaguely,
the event mechanism and all of the interfaces in the package
\texttt{CH.ifa.draw.framework}
before digging deeper into the sources.

\section{Hierarchy \texttt{de.renew}}

This hierarchy constitutes the Petri-net-specific part of Renew.
Two subhierarchies can be found here: \texttt{de.renew.formalism}
and \texttt{de.renew.gui}, which are responsible for implementing
specific net formalisms and for implementing a GUI, respectively.
But first we discuss the other packages in this hierarchy,
which constitute the core simulation engine.

\subsection{Simulation Engine}

These packages are concerned with the non-graphical representation
and execution of high-level Petri nets.

\begin{description}
\item[\texttt{de.renew.application}]
  This package contains applications that allow the non-graphical
  stand-alone simulation of Petri nets. At the moment,
  \texttt{ShadowSimulator}, which bases the simulation
  on a serialized shadow net system, is the only
  class here. Other applications might follow.
\item[\texttt{de.renew.call}] 
  This package supports nets that implement Java methods. This is done by
  automatically generating subclasses of 
  \texttt{de.renew.simulator.NetInstance}, so-called Net Stubs, that convert
  method calls to synchronous channel invocations.
  These invocations are subsequently stored in the usual search queue
  and executed by the simulator as soon as possible.
  The Renew User Guide contains a description of how to utilize Net Stubs.
\item[\texttt{de.renew.event}] 
  This package contains the event mechanism that is used to couple
  the simulator and the GUI.
  It provides interfaces for event listeners and producers and
  an abstract convenience implementation for listeners.
\item[\texttt{de.renew.function}] 
  This package contains a variety of function objects. An implementation
  of the \texttt{Function} interface must provide a single method
  that converts its argument into a result.
  There are predefined function objects to perform all of the transformations
  of objects that can be specified in a Java expression. This
  package is not yet concerned with variables or assignment,
  but only with operations on runtime objects and values.
\item[\texttt{de.renew.shadow}]
  The shadow layer separates the GUI from the
  execution layer. Shadow nets represent the net drawings, 
  but they abstract from all information that does not influence
  the simulation, like position, size, or color of the net elements.

  This package is based on a create/discard API: You can create
  net elements, but few changes can be applied to these elements
  without discarding and recreating them. In fact, even discarding
  shadow nets is discouraged. Instead, an application should create
  a new shadow net each time this is required.

  The implementations of \texttt{ShadowCompiler} are responsible for
  creating nets at the semantic layer using the shadow net.
\item[\texttt{de.renew.remote}] 
  This package allows to access the state of a running simulation
  remotely via RMI calls. In some sense, it will be
  the counterpart of the shadow package. While the \texttt{de.renew.shadow} 
  package allows it to convert net drawings to nets, this
  package will facilitate the display of net instances in net instance
  drawings.
  \newonedotsix{%
    This package is now fully functional. The gui packages do no
    longer access simulator classes directly to display a
    running simulation. Instead, all state information is
    obtained through remote accessor objects.
  }
\item[\texttt{de.renew.unify}] 
  This package encapsulates a unification algorithm.
  Unification is done in-place using modifiable objects.
  The unification algorithm supports a backtracking mechanism
  using the class \texttt{StateRecorder}.
  Ordinary Java objects are unifiable
  if and only if they are equal according to \texttt{Object.equals(...)}.
  Variables can send notifications after they become fully bound,
  so that functions can be evaluated as soon as their arguments
  are fully known. An occurrence check is performed.
  The special class \texttt{Calculator} of
  unifiable objects is unifiable only with itself, but is
  incorporated into the occurrence check. These objects are 
  used to represent calculations that can be based on some objects
  and might lead to an arbitrary result.
\item[\texttt{de.renew.util}] 
  A package of miscellaneous classes that are used somewhere else,
  but were considered too useful across applications.
\item[\texttt{de.renew.expression}] 
  This package combines the functionality of
  the \texttt{de.renew.function} package and
  the \texttt{de.renew.unify} package and implements
  full Java expressions. It adds those expressions
  that cannot be interpreted by a mere function evaluation.
  Especially, this package handles the introduction of local variables.
  To do this, every expression is evaluated in the context of a
  variable mapper, which maps variable names to variables of the
  unification mechanism. Other additions of this package are
  type checks, constants, and bidirectional expressions.
\item[\texttt{de.renew.engine.common}] 
  This packages defines some reusable occurrences and
  executables. These objects are not sufficient for
  defining a formalism, but they constitute essential
  elements for most formalisms.
\item[\texttt{de.renew.engine.searcher}] 
  This package contains the basic algorithms that are
  used during the search for an activated binding.
  Classes and interfaces remain very abstract here:
  We talk about things that may be searched and things
  that may be executed, but do not give any concrete
  examples.
\item[\texttt{de.renew.engine.searchqueue}] 
  This package defines the \texttt{SearchQueue}, which is responsible
  for keeping track of possibly enabled \texttt{Searchable}s so
  that they can later on be searched for activated binding.
  The search queue is also keeping track of the current simulation
  time, so that it may order the searchables according to the earliest
  possible time when a search may possibly succeed.
\item[\texttt{de.renew.engine.simulator}] 
  A simulator makes sure to retrieve searchables from the
  search queue and to search them for activated binding.
  the binding are then executed with a policy suitable for
  the simulator, be it concurrently or sequentially.
\item[\texttt{de.renew.net}] 
  This package defines the basic building block for defining nets
  and net instances: places and transitions. It does not yet deal
  with arcs and transition instances.
\item[\texttt{de.renew.net.arc}] 
  Here we define various arc types for use with nets.
\item[\texttt{de.renew.net.event}] 
  Events and event listeners are used when dynamic chnges of
  a net's state and of actions in the net must be tracked
  by other code.
\item[\texttt{de.renew.net.inscription}] 
  Transition inscriptions augment a transition's behavior.
  The inscriptions defined in this package are by no way
  exhaustive, but represent the most commonly used inscriptions.
\item[\texttt{de.renew.database}] This package interacts with the
  simulator core and makes sure that all changes in all nets
  can be recorded in a database. The access to the database is
  transactional, i.e., the database driver is notified about all
  tokens that are moved by a single transition in a single
  method call and is supposed to record these changes permanently
  and atomically.
\item[\texttt{de.renew.watch}] This package allows to keep track of all
  possible valuations of a synchronous channel of a net instance.
  Afterward, one valuation may be chosen to request an explicit firing
  of a transition.
  This has been applied to design a workflow engine, but other
  uses are imaginable, as it allow an external entity to watch and
  control the flow of the simulation.
% \item[\texttt{de.renew.test}] This package provides a few
%   automated test scenarios, which aid in maintaining the correctness
%   of some crucial parts of Renew. At the moment, there are not
%   many tests available, but it is planned to extend this
%   package considerably.
% \item[\texttt{de.renew.debug}] This package contains a few
%   miscellaneous classes that turned out useful during the
%   implementation of Renew. It is not required to use or 
%   understand this package.
\end{description}

\subsection{Net Formalisms}

\begin{description}
\item[\texttt{de.renew.formalism.java}] This package 
  provides a compiler for the Renew default net formalism
  that inputs shadow nets and outputs nets at the semantic layer.
  Most other net formalism are based on this package.
  Typically, only the parser needs to be exchanged for a different
  one in order to accommodate for a different syntax.
\item[\texttt{de.renew.formalism.stub}] A variant of the
  Java net formalism that experiments with some improved
  type rules, which are essential when generating net stubs.
  Usage of this net formalism is discouraged.
\item[\texttt{de.renew.formalism.bool}] A special
  net formalism that implement Boolean Petri nets.
  These nets are especially useful to experiment with workflows
  that are derived from event-driven process chains.
\item[\texttt{de.renew.formalism.fs}] This formalism was a first
  prototype to include Feature Structures (see below) into Renew.
  It is now only used as a basis for the following formalism and
  for debugging.
\item[\texttt{de.renew.formalism.fsnet}] This formalism 
%has been   developed in the PhD thesis of Frank Wienberg and
  supports Feature Structures as a means to describe object
  constraints.
  It includes a type modelling tool to easily specify
  used defined types. Also, all available Java classes can be
  used as types. Tokens are Feature Structures which can be
  combined using unification and can be tested for subsumption.
\end{description}

\subsection{Graphical User Interface}

\begin{description}
\item[\texttt{de.renew.gui}] This contains the GUI
  for drawing, editing, and simulating Petri nets with Renew.
  Specialized figures aid in the presentation of the net
  and new tools modify these figures. The application class
  \texttt{CPNApplication} is responsible for setup and
  coordination.
\item[\texttt{de.renew.gui.xml}] This package contains
  an experimental XML storage format for nets. The XML
  import function uses the visitor pattern and requires a SAX-compatible
  XML parser.
\item[\texttt{de.renew.gui.maria}] This package features a special
  mode for drawing nets of the Petri net analyzer Maria.
  No simulation support is available, but it is possible to design
  a net graphically and export it to simulation in Maria.
\item[\texttt{de.renew.gui.fs}] This package provides the GUI
  elements needed in the FSNet formalism (see above).
  These are type modelling figures and Feature Structures.
  The Feature Structure figure is also used to render
  Java objects as \texttt{expanded Tokens}.
\end{description}

\section{Package \texttt{de.uni\symbol{95}hamburg.fs}}

This package is an implementation of a typed Feature Structure
formalism and has been developed for Renew's \texttt{FSMode}.
A Feature Structure is a graph with typed nodes and labelled arcs.
The arc labels can be seen as features or attributes of the nodes.
A Feature Structure points to a root node of such a graph.
The type system allows subtypes (\emph{is-a}-relations) and
incompatible types (\emph{is-not-a}-relations).
Types can be tested for equality and subsumption.
Among the functions available for Feature Structures is subsumption
and unification, which is in fact graph unification.
The packages \texttt{de.renew.formalism.fs},
\texttt{de.renew.formalism.fsnet}, and \texttt{de.renew.gui.fs}
depend on this package.

% Local Variables:
% TeX-master:"architecture.tex"
% End: