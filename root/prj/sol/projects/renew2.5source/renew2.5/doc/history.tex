\section{History}\label{sec:history}

Version 1.0 was the first public release. It
included a net editor, a reference net
simulator, a Java stub compiler, and example nets.

\subsection{Changes in Version 1.1}

\subsubsection*{Modifications}

Some performance enhancements were implemented and minor
bugs were fixed. Some source level inconsistencies were
cleaned up.
The thread model of Java 1.2 was adopted.
The source code was changed to be compilable with Java 1.1.3.
The windowing code was made more robust under Java 1.2.

The handling of \texttt{null}-objects in the simulator was corrected.
The type system was made more compatible with the Java type system.
The trace flag of netelements is now saved to disk.
The simulation performance was improved.
The garbage collection of net instances was improved.

The graphical user interface was improved for some window managers.
The presentation of current markings was improved.
The interactive execution of reference nets has been
improved a lot (see Section~\ref{sec:simulation}).

\subsubsection*{Additions}

The parallel simulation code was added.
The checks for double names and for cyclic channel
dependencies were added.
Transition inscriptions may now include several parts
separated by semicolons.
Virtual places may now be used in nets.

During the simulation, bindings can be selected and fired
under user control.
The multiset of tokens contained in
a place instance can be displayed as just the cardinality of the
multiset, a collection of all tokens in the multiset directly
within the drawing, or in a separate window.
Individual components of tuples can
be inspected. Initial markings are hidden during the simulation.

\subsection{Changes in Version 1.2}

\subsubsection*{Modifications}

The simulation engine was made more robust and flexible.
Minor bugs were fixed.

A single inscription figure may now contain multiple
arc inscriptions or initial marking inscriptions that are separated
by semicolons.
Slight inconsistencies in the inscription language were
cleaned up. The type rules were improved.
The results of action inscriptions may now be passed through
synchronous channels even in the presence of typed variables.

Some display problems with Java 1.2 have been fixed.

\subsubsection*{Additions}

Flexible arcs were added.
Clear arcs were added.
Inhibitor arcs were added.

Marked places and firing transitions can now be highlighted during
the simulation.
A rudimentary net layout algorithm has been implemented.
The state of a running simulation can now be saved and restored.
Restarting a simulation may now reload Java classes.

Export of Encapsulated PostScript was implemented.
Selection of groups of figures was improved.

\subsection{Changes in Version 1.3}

\subsubsection*{Modifications}

Some minor improvements of the graphical user interface were
applied.

\subsubsection*{Additions}

The timed simulation mode was added.
Lists were provided in addition to tuples.

Breakpoints were added in order to control the
graphical simulation.
An XML import and export facility was added.
A graph layout algorithm that may help in viewing nets
was added. More commands for arranging figures manually were
provided. The ability to select and deselect figures by type
was added.

\subsection{Changes in Version 1.4}

\subsubsection*{Modifications}

This was a maintenance release that provided mainly
improvements in the user interface, documentation updates, and
bug fixes.

\subsubsection*{Additions}

You can now inspect token Java objects in detail and
put toolbars into their own window. You can
insert intermediate points into connections more easily.

\subsection{Changes in Version 1.5}

\subsubsection*{Modifications}

The compatibility with Java 1.2 was improved.
Bugs in the simulation engine were fixed.
Some GUI problems were corrected.
The menu structure was cleaned up and simplified.

\subsubsection*{Additions}

A persistent database backing supports the deployment 
of the simulator in environment with high availability 
requirements.

Arcs can be B-splines. 
An alignment can be specified for every text figure.
Transitions and places can be refined.
Subnets can be coarsened. Nets can be merged.

Drawings are now autosaved. A backup copy of every file is kept.
Undo and redo commands were added. Search and replace commands
were added.

The architecture guide was added, which is a manual that
describes the most important internal algorithms
and data structures of Renew.


\subsection{Changes in Version 1.5.1}

\subsubsection*{Modifications}

This was a maintenance release that provided
bug fixes for the simulation engine.

\subsection{Changes in Version 1.5.2}

\subsubsection*{Modifications}

This was a maintenance release that provided
bug fixes for the install scripts and a performance improvement
of the simulation engine.


\subsection{Changes in Version 1.6}

\subsubsection*{Modifications}

Java 1.2 is now required for compiling and running Renew.
Bugs in the Java net parser, class loader and simulation engine
were fixed.
Shadow net system serialization and rendering has been fixed.
The windows menu is sorted alphabetically.
Windows can be de-iconified.

\subsubsection*{Additions}

A remote layer allows the separation of the user interface from the
simulation engine.
A net loader allows on-demand loading and compilation of nets
during a simulation.

A new transition inscription \texttt{manual} was introduced
for transitions that are not supposed to fire automatically
in ordinary running simulations.


\subsection{Changes in Version 2.0}

\subsubsection*{Removals}

The \texttt{RenewMode} interface provided by the GUI has been removed.
In consequence, the start scripts for the modes disappeared, too.
The channel, name and isolated node checks have been removed from the
\texttt{Net} menu because they need to be adopted to the new
simulator architecture.

\subsubsection*{Modifications}

Java 1.4 is now required for compiling and running Renew.
The application was decomposed into several plug-ins.
The simulation engine was restructured.
The GUI application classes were restructured and (partly) converted
to use the Swing package from the Java foundation classes.
The import and export menus have been restructured.
The handling of the various configuration properties has been
canonised by the plug-in system.
The class loader for custom classes has changed.

The \texttt{:net()} channel is no longer invoked implicitly on
instance creation.
The class \texttt{SequentialSimulator} is replaced by the
\texttt{NonConcurrentSimulator} without deadlock detection
feature.
The expanded token display feature has moved into the optional FS
plug-in, this option has no effect unless the FS plug-in is
installed.

\subsubsection*{Additions}

A plug-in system was added as bottom layer of the application.
The ability to switch simulator modes, net formalisms, the net loader
path, and the remote access feature on the fly was added.
A PNML-compatible export format was added.
The editor is able to load drawings from URLs.
The net loader can now search nets relative to the classpath.

\subsection{Changes in Version 2.0.1}

\subsubsection*{Modifications}

This was a maintenance release that provided
bug fixes for the install scripts 
and some redraw issues of the graphical net editor.

\subsubsection*{Additions}

An experimental \texttt{AppleUI} plug-in is available as optional
download. It provides rudimentary integration with the Mac OS look\&feel.

\subsection{Changes in Version 2.1}

\subsubsection*{Modifications}

Many error messages of the Java net compiler or about problems in a running
simulation became more detailed.
The command-line tool \texttt{ShadowTranslator} and the corresponding Ant
task now optionally include the specified formalism and a syntax check.
Fixed transition modes of bool net compiler.
Fixed manual transitions in saved simulation states.

The color and font attribute dialogues were improved.
Whitespace-only inscriptions are now deleted automatically (like empty
ones).
The GuiPrompt plug-in now provides a text area for command feedback.
The binding selection frame is now scrollable.
Tool windows and dialogues are now listed in the \texttt{Windows} menu.
Breakpoints pre-set via the \texttt{Net} menu are now visually tagged.
Fixed some drawing edit bugs in the GUI.
Fixed some rare deadlocks in the token game display.
Improved scrolling effect of mouse wheel in drawings.
Fixed handling of polygons.

Some important changes to configuration properties are documented in the
upgrade notes (see Section~\ref{sec:upgrade}).
Developers might also have a look there because of some code changes.

The set of Ant build files that come with the Renew source has undergone
some changes.
The build process now stores information about the compilation environment
with the plug-ins.
Distribution file names can now optionally include version information.
The \texttt{list} and \texttt{info} commands optionally display this
information.

\subsubsection*{Additions}

All Renew components (except the console prompt) now use the Apache Log4J
logging framework instead of Java console output. 
In the default configuration, informational and error messages are printed
to the console and logged to a file.
The simulation trace also goes to the logging framework (see
Section~\ref{subsec:log4jConfiguration}).
The new Logging plug-in provides a simulation trace window within the GUI.

The \texttt{Net step} option has been added to the simulation menu.

\subsection{Changes in Version 2.1.1}

\subsubsection*{Modifications}

This was a maintenance release that provides several minor bug fixes for
PNML export, \texttt{null} token display and access to public methods of
private (inner) classes in Java expressions.
In addition, this release is capable of reading drawing files created with
the later release 2.2 (with some minor exceptions).

\subsubsection*{Additions}

The AppleUI plugin now supports building a Mac OS X application bundle.

\subsection{Changes in Version 2.2}

\subsubsection*{Modifications}

Java 1.5 is now required for compiling and running Renew.

The Gui now uses the Graphics2D Framework that came with Java 1.2, so some of the figures might
be drawn a little bit different when it comes to size and style.
Tokens are now displayed on a white opaque background in the token game to
increase readability.
Scrolling now continues if the mouse is moved outside a drawing while a
button is pressed.

Modifier keys (Ctrl, Shift) have been added to several commands and tools
on drawing figures.
These enable users to resize figures to equal width and height, to adjust
polygon vertices at right angles with their adjacent edges, and to restrict
polygon transformations to either scaling or rotation.
Keyboard movement of figures with arrow keys can now be sped up using the
Shift modifier.

The behavior of Search and Replace has been fixed so that multiple instances of the search string 
in the same figure are now found and replaced correctly.
The Drawing Load Server has been restricted to accept local connection
requests only.

\subsubsection*{Additions}

Renew now includes and uses the VectorGraphics packages of the FreeHEP
project to accomplish graphical export of drawings. 
Additional supported export formats are PDF, SVG, and PNG.
EPS export now exports non-standard fonts correctly (at the price of larger
files).
EPS files now always have a rectangular white canvas.

A new pie figure allows to draw segments of arcs and ellipses.
Line styles (dotted, dashed, etc.) can now be applied to boxes, ellipses
and other figures with outlines.
A transparency attribute has been added to all figure, font and pen colors.
This breaks compatibility with older Renew versions, so that drawings saved with version 2.2 
can not be opened by previous versions (except release 2.1.1).
The transparency attribute is currently ignored when exporting drawings to EPS and this feature 
might not be implemented in future versions.
For drawings with transparency use one of the two new export formats SVG or
PDF, which handle transparency correctly.

The hotkey Ctrl+M now brings the menu and toolbar frame to front.
Added ``show net pattern/instance element'' options to the context menu in
the simulation trace window.
The net stub compiler now additionally supports stub objects that wrap
themselves around an existing net instance during instantiation (before, a
stub object always created its own net instance).

We provide a Mac OS X application bundle as well as configuration files for
the FreeDesktop (e.g. Gnome) environment that allow desktop integration
with separate icons and mime-types for Renew document files.
However, there is still no such support for the Windows family of operating
systems.

\subsubsection*{Relevant for developers only}

GUI and simulation have been separated so that they use different threads now. 
All calls to the simulation are decoupled and executed in specialized
simulation threads.
All calls to the GUI are delegated to or synchronized with the AWT event
thread.
Simulation threads can now be configured with a separate priority.
Loading of user-supplied classes in the context of simulations has been improved.

A new parameter ``netpath'' has been added to the Ant task to create shadow
net systems.
The Ant build environment has been enhanced to support separate source
code trees for JUnit tests and Cobertura coverage reports.
However, there still are nearly no test cases implemented.
Several tools that form the Renew build environment are now required in newer releases.
Please refer to the readme file in the source package.

\subsection{Changes in Version 2.3}
\label{sec:changes-2.3}

\subsubsection*{Modifications}

Java 1.6 is now required for compiling and running Renew.

Renew now includes and uses the 2.2 version of the FreeHEP project for
graphical exports of drawings.

Renew offers a better syntax check for Java reference net
models. If a Java inscription references a non-existing method or
field of an object, a proposal for existing methods or fields is
made instead of just pointing out the syntax error (see also
Section~\ref{sec:errors}).

Minor modifications to the graphical editor functionality of Renew
consist of the following. The names and colors of place figures are
now transferred to their virtual places. The editor prevents adding
more than one arc inscription by right-clicking on an arc with the
mouse as this happened rather by accident than on purpose. However, it
is still possible to add multiple arc inscriptions by using the
inscription tool.

\subsubsection*{Additions}

On startup Renew displays a splashscreen that gives information about
the loaded plugins.

There are two new entries in the File menu. The first addition is a
list of recently saved drawings. The second addition is the
possibility to open the \emph{Renew Navigator}, which allows to import
file folders and show their content in a tree view. A more detailed
description of how to use the navigator can be found in
Section~\ref{sec:navigator}.

It is now possible to define re-usable \emph{Net Components}. A net
component consists of a set of net elements that typically fulfill
some generic function and can be treated as a whole in a larger net
model. More details can be found in Section~\ref{sec:net-components}.

The background of expanded tokens in instance/simulation
drawings can be changed to be transparent by setting the property 
\texttt{de.\allowbreak{}renew.\allowbreak{}gui.\allowbreak{}no\allowbreak{}Token\allowbreak{}Background}.

Several keyboard shortcuts have been changed and more have been
added, especially for selecting the main drawing tools. A
comprehensive list of existing shortcuts can be found in
Appendix~\ref{sec:keyboard-shortcuts}.

\subsubsection*{Relevant for Developers only}

Generics are now used throughout the code.

The RMI functionality which was formerly included in the Simulator
plugin was extracted into a new Remote plugin.

The lock functionality was moved from the Simulator to the Util
plugin.

% Some bugs have been fixed in this version:
% \begin{itemize}
% \item Resolved deadlock in SimulatorPlugin facade (Bugzilla bug \#222).
% \item Fixed bug in handling boolean properties.
% \item Fixed handling plugin locations which had special characters in their filenames.
% \item Plugins can now add reference nets from multiple jar files
%   (previously only files from last jar file were loaded).
%   \TODO{mwe: verstehe ich nicht}
% \end{itemize}

There are several changes to the Ant build environment. The Ant target 
\texttt{clean} in the meta build file now iterates over
all subdirectories instead of having a fixed list of plugins. Source
files of nets (.rnw) can optionally be included in the generated
plugin archives (.jars) with the Ant target \texttt{rnw}. To activate
this function you need to set the property
\texttt{option.include.rnws} in your Ant properties (build.xml of the
plugin in question or ant local.properties). The property
\texttt{option.sns.compile} switches the syntax check for shadow net
files (.sns) on and off. The Ant target \texttt{javac} accepts an
encoding parameter which is set via the property
\texttt{option.compile.encoding} and defaults to utf-8. The Ant task
\texttt{createpng} allows to export net drawings to .png files.

%  Another change is related to the loading mechanism for plugins.
% \TODO{describe loading\\}

\subsection{Changes in Version 2.4}

\subsubsection*{Modifications}

The support for the \emph{.xrn} format is discontinued. We encourage the use
of PNML instead.  
We fixed the remote server connection (RMI) by providing
configuration (see the User Guide Section 2.6.)  
We have fixed the
simulation database backing and adapted the mechanism for MySQL with InnoDB.  
Remaining AWT dialogues have been converted to Swing.  
The Logging GUI has been improved by decoupling the Simulator from the GUI.  
The \emph{loadrenew} script now starts a regular Renew instance if the
connection to a Drawing Load Server is not possible.  
The desktop integration for Linux and Windows have been improved. We provide new unified icons for all operating systems.

Many minor bugs
have been fixed. Some of these were:
\begin{itemize}
\item Rare problems with locating nets relative to the \emph{classpath} have been
  solved.
\item Changes on the line style now affects all figures.
\item The font style \emph{underlined} now also affects small font sizes.
\item Export to PNML now always produces files in UTF-8 character
  encoding.
\item The Logging tab of the Configure Simulation dialog has been
  revised.
\item The log4j PatternLayout can now be set from within the GUI
\item Custom file appenders created in Logging GUI are now functional.
\item Net components are more robust if attached figures are manipulated.
\item It is now possible to escape whitespaces in command line commands. 
In that way it is possible to open drawings with whitespaces in the path 
from command line.
\end{itemize}

\subsubsection*{Additions}

The Navigator now offers a button to recursively expand a directory
sub-tree.  The Navigator now loads directories without locking the
GUI. PNML and ARM files are shown in the Navigator.  
The keyboard
shortcut \emph{Ctrl+Enter} closes the text editor overlay.  

The background transparency of EPS files can now be controlled by
setting the property \texttt{de.renew.io.export.eps-transparency}.
For Windows, the installation script \emph{installrenew.bat} creates reg
 files that associate and disassociate Renew drawing files with 
 \emph{loadrenew.bat} and register icons.  
 We provide deb packages for Debian-based systems.


\subsubsection*{Relevant for Developers}


We refactored large parts of the code base.  Many Java compiler
warnings have been resolved and Javadoc documentation have been
improved.  A few more JUnit tests have been introduced.  Logging and
the simulator have been partially reworked to allow deadlock-free
real-time (GUI) logging with only minimal time delay.  Several tools
that were originally mandatory to build Renew are now optional: these
are Cobertura, Jalopy, JUnit and Latex.

\subsection{Changes in Version 2.4.1}

This is a maintenance release that provides a fix for a race condition
that occurs - under rare conditions - during the termination of the
simulation.

% see https://tgiprojekte.informatik.uni-hamburg.de/issues/1527 for details

\subsection{Changes in Version 2.4.2}

This is a maintenance release that provides a fix for the import
of reference nets from PNML format (RefNet).

% see https://tgiprojekte.informatik.uni-hamburg.de/issues/3259 for detail

\subsection{Changes in Version 2.4.3}

This is a maintenance release that provides an update of the FreeHEP library
and a new version of the Mac OS X application bundle.
It fixes issues concerning the export functionality with newer Java versions.
This version requires at least Java 7.

% see https://tgiprojekte.informatik.uni-hamburg.de/issues/4327 for details

\subsection{Changes in Version 2.5}

\subsubsection*{Modifications}

We have modified several features of Renew.
Most obvious is the complete reimplementation of the Navigator plugin. 
It is now persistent, extensible and the tree view can now be filtered.
We optionally provide some convenient extensions, 
such as the integration of the drawing's diff feature (ImageNetDiff), 
which can now be triggered directly from the Navigator GUI.
The FreeHEP library has been upgraded to version 2.4.
The quick-draw\footnote{The possibility to quickly draw an arc and a n
  node by using the arc handle (see
  Section~\ref{subsubsec:toolTransition}).}  feature has been
improved, which results in a reduction to half the amount of mouse
clicks during quickly drawing net elements.
Some key-bindings are now configurable.

The log4j configuration and the configuration GUI have been improved. 
The individual log4j configuration file now resides in folder \emph{.renew}, 
which is located in the users home folder. The default location for log files moved there, as well.
The loading of plugins at startup can now be black-listed or white-listed.
The PDF export produces  PDF documents with bounding boxes; configuration has been fixed.
The grid can now be adapted and it can be activated as default.
Several console commands have been improved, including the
following plugin commands: \emph{list}, \emph{load},
\emph{unload} and also  export command \emph{ex}. Type \emph{help}
to print a synopsis of all commands.

\subsubsection*{Additions}
The Console plugin replaces the Prompt plugin. 
It employs the well-established JLine2 libray and provides several improvements:
tab-completion for commands and attributes, command history and editable command line.
The Quick Fix feature improves the reporting of syntax errors by
providing suitable proposals for remedies and their automatic
realization.
The Refactoring plugin (optional) provides features such as renaming
of variable or renaming of synchronous channels.
Drag \& drop now works for Renew drawings. Simply pull the drawing
over Renew's main window. Drawings and also folders can also be
dragged into the Navigator Window. Additionally you can add images to
a drawing by dragging it on the drawing's window.
Two new text tools have been added. One is the target text tool, which
allows to add hyperlinks to any drawing element. The targets can be
other model artifacts, for instance a net, or external resources
referenced by a URL. The hyperlink is activated by using the
\emph{Ctrl} modifier key together with a mouse click.
The comment tool allows to quickly add comments to a drawing element.
    
\subsubsection{Removals}
\begin{itemize}
\item Macao format has been removed, since its usability was very limited.
\item The PostScript export has been removed. Use EPS export or PDF export instead.
\end{itemize}



% Local Variables:
% mode: latex
% TeX-master:"renew.tex"
% End:
