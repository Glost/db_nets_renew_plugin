\chapter{Introduction}

The intention of this document is to give the interested user of Renew
some rudimentary insight into the general structure
and some individual components of Renew. This document does not aim
to be a programming guide, but it tries to help you in
finding a point where to start.

Note that, in general, you do not need to read this document.
You can work with Renew perfectly without knowing its internals.
Whenever you have a problem that does not seem to be solvable
with the current version of Renew, think once more. If you still
consider an extension of Renew essential, you should start with this manual
before digging into the source.

To gain something from reading this manual you need considerable
Java experience, the Renew source package, and willingness to read
the source. Although large parts of the source are not yet documented,
you might be able to infer most of the functionality from the
names given to classes and methods, especially after checking this manual.

\section{Customizing Renew}

Because Renew is available with source, it is tempting to
create a customized version of it. This is indeed possible
and this manual will help you with the task,
but there are a few drawbacks:
\begin{itemize}
\item Currently Renew is only partially commented. 
\item You diverge from the main development line. The internal structure
  of Renew might change significantly in future versions. If you send
  your additions/improvements to us,
  we will try to include them in the main release.
\item This internal programming guide still needs considerable
  work to cover all aspects of Renew.
\end{itemize}
However, there might be reasons to modify Renew due to your
particular applications.

If you implement a new net formalism, we suggest that you start a
new Java package for it, e.g.\ \texttt{de.renew.evenbetter}.
You might want to place the package in your own package hierarchy.

Not all customization can be done by simply adding more classes.
Sometimes you might require some changes to the standard classes.
It is a good idea to check such issues with the Renew team,
because sometimes we might know best where to place
this hook or that abstraction.

Although we do not require this as part of our license,
we will be very happy if you send us any modified
source code.



\section{Before you Start}

Before you start reading this manual, you should run JavaDoc to create
the online documentation from the source files. This will help you
in getting a feel of the system and in following this manual
more easily. Many classes in the \texttt{de.renew}
hierarchy do not contain JavaDoc comments, but you will still find it
valuable to browse through the classes easily.

Since Renew uses the Java library, you should have learned
everything there is to learn about the packages
\texttt{java.lang} and \texttt{java.util}. You will not be able to
proceed without that knowledge.

Since Renew uses Doug Lea's collection classes \cite{Lea98},
you should familiarize yourself with this class library before
looking at the simulation engine.
The original distribution has some API documentation included.

Since Renew uses the Reflection API to execute Java inscriptions,
you should familiarize yourself with the package
\texttt{java.lang.reflect} before
looking at the simulation engine.
See the original documentation from Sun for details.

Since Renew uses the Java AWT for its windowing code,
you should consult at least a tutorial about the
package \texttt{java.awt}.

Since Renew uses the JHotDraw graph editing framework \cite{Gamma98}
for painting its nets, you should inform yourself about this
package before investigating the GUI code. The classes in the 
\texttt{CH.ifa.draw} hierarchy contain nice JavaDoc comments
and are a good place to start.

If you want to customize the inscription grammar, you should
download JavaCC, the Java Compiler Compiler, from
\url{http://javacc.dev.java.net/}
which is free.
We consider doing a complete rewrite of the grammar files
in order to free them of the license restrictions that
currently exist. In that course of work, it might be
sensible to move from JavaCC (a great program) to ANTLR 
(a great program that is also free and available with source).

Class diagrams in this document are given in UML style notation.
You are encouraged to learn about this notation from one of the many
books or directly at \cite{rational}. In all diagrams we aim at
providing the best overview and structural knowledge. To this end,
the diagrams do not always include information about all attributes, 
but rather those attributes that are useful for understanding that
part of the architecture that is currently discussed.

Sometimes we use attributes where one might rather expect
an association or vice versa, but this is always done in order
to simplify the diagram and to convey the intended meaning of a construction.
For similar reasons, private attributes 
that are publicly accessible via getter and setter methods
are sometimes shown as public attributes.


\section{Overview}

We will now explain the basic structure of this document.
In Chapter~\ref{chap:overview} we give an overview of the packages
that structure the Renew application.



\section{Acknowledgements}
We would like to thank Prof.\ Dr.\ R\"udiger Valk and Dr.\ Daniel Moldt
from the University of Hamburg for interesting discussions, help, 
and encouraging comments. 

We would also like to thank 
S\"onke R\"olke,
Dennis Schmitz and
Martin Wincierz
for their work during the preparation of this
release.
We would like to thank J\"orn Schumacher
for the prototype of the plug-in system (2.0), Benjamin Schleinzer
for his work during the preparation of former releases (2.1-2.2) and 
Berndt M\"uller who %%Farwer 
has been of great help with respect to previous
Renew releases for Mac OS ($\leq{}$2.0).
Some nice extensions of Renew were suggested or programmed
by Michael K\"oh\-ler and Heiko R\"olke.

We are indebted to the authors of various freeware libraries, namely
Mark Donszelmann, %%% freehep vectorgraphics
Erich Gamma, %%% JHotDraw
Doug Lea, %%% collections package (still used in FS)
David Megginson, %%% SAX
Bill McKeeman %%% Java Grammar
and Sriram Sankar. %%% JavaCC
%Marc Prud'hommeaux (JLine1), Jason Dillon, Guillaume Nodet, Colin Jones (JLine2)

Dr.\ Maryam Purvis, Dr.\ Da Deng, and Selena Lemalu 
from the Department of Information Science
(\texttt{http://infosci.otago.ac.nz/}), University of
Otago, Dunedin, New Zealand,
kindly aided us in the translation of parts of the documentation
and are involved in an interesting application project.

Valuable contributions and suggestions were made by
students and scientific workers at the University of Hamburg, most notably
Hannes Ahrens,
Tobias Betz, 
Jan Bolte,
Lars Braubach,
Timo Carl,
Dominic Dibbern,
Friedrich Delgado Friedrichs,
Matthias Ernst, 
Max Friedrich,
Daniel Friehe,
Olaf Gro\ss ler, 
Julia Hagemeister,
Sven Heitsch,
Marcin Hewelt,
Jan Hicken,
Thomas Jacob, 
Andreas Kanzlers,
Lutz Kirsten,
Michael K\"oh\-ler, 
Till Kothe,
Annette Laue, 
Matthias Liedtke, 
Marcel Martens, 
Klaus Mit\-rei\-ter,
Konstantin M\"ollers,
Eva M\"uller,
Jens Nor\-gall,
Sven Offermann,
Felix Ortmann,
Martin Pfeiffer, 
Alexander Pokahr,
Tobias Rathjen,
Christian R\"oder,
Heiko R\"olke, 
Benjamin Schleinzer,
Jan Schl\"uter,
Marc Sch\"on\-berg,
J\"orn Schumacher,
Michael Simon,
Fabian Sobanski, 
Volker Tell,
Benjamin Teuber,
Thomas Wagner,
Matthias Wester-Ebbinghaus, 
and Eberhard Wolff. 



We would like to thank the numerous users of Renew who provided
hints and constructive criticism. They helped greatly in improving the
quality of the code and the documentation. In particular, 
we would like to name Alun Champion and Zacharias Tsiatsoulis.

%%% Local Variables: 
%%% mode: latex
%%% TeX-master: "renew.tex"
%%% End: 



