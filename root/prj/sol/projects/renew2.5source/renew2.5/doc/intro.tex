\chapter{Introduction}

On the following pages, you will learn about Renew, the
Reference Net Workshop. The most important topics
are:
\begin{itemize}
\item installing the tool (Chapter \ref{ch:install}),
\item the reference net formalism (Chapter \ref{ch:reference}),
\item using Renew (Chapter \ref{ch:usage}),
\end{itemize}
Both reference nets and their supporting tools are based
on the programming language Java. To be able to use them
to their full capacity, some knowledge of Java is required.
While the basic concepts of Java will be explained in this document,
there are plenty of books that will serve as a more in-depth
introduction to Java. \cite{vdLinden96} is a good first start for 
experienced programmers.

If you encounter any problem during your work with Renew,
we will try to help you. See Appendix~\ref{ap:contact}
for our address. At the same address, you can make suggestions
for improvements or you can request information on the latest
release of Renew. If you want to submit example models or
extensions to the tool, that would be especially welcome.


\section{Should I Use Renew?}

The main strength of Renew lies in its openness and versatility.
\begin{itemize}
\item Renew has been written in Java, so it will run on all
major modern operating systems without changes.

\item Renew comes complete with source, so its algorithms may 
be freely extended and improved. It is in fact possible
to add special net inscriptions quickly. It is even possible
to implement completely new net formalisms without changing
the basic structure of Renew.

\item Renew can make use of any Java class. Today there exist
Java classes that cover almost all aspects of programming.

\item Reference nets are themselves Java objects. Making
calls from Java code to nets is just as easy as to make calls
from nets to Java code.
\end{itemize}
The Petri net formalism of Renew, too, might be very interesting for
developers.
\begin{itemize}
\item Renew supports synchronous channels.
Channels are a powerful communication mechanism and they can be used as
a reliable abstraction concept.

\item Net instances allow object-oriented modeling with Petri nets.
While a few other net formalisms provide net instances, it is their 
consistent integration with the other features that makes them useful.

\item Reference nets were specifically designed with garbage
collection of net instances in mind, which is indispensable for
good object-oriented programming.

\item Many arc types are available that cover almost all
net formalisms. Simulation time with an earliest firing time
semantics is integrated.
\end{itemize}
There are, however, a few points to be aware of.
\begin{itemize}
\item There are currently only rudimentary analysis tools for Renew. 
Although a few export interfaces have already been implemented,
useful analysis seems a long way off.
Currently, Renew relies entirely on simulation to explore the
properties of a net, where you can dynamically and
interactively explore the state of the simulation.

However, for many applications, analysis does not play a
prominent role. Petri nets are often used only because of
their intuitive graphical representation, their expressiveness,
and their precise semantics.

\item During simulation, the user cannot change the current 
marking of the simulated net except by firing a transition.
This can make it somewhat more difficult to set up a desired 
test case.

\item In our formalism, there is no notion of firing probabilities
or priorities. By exploiting the open architecture of Renew,
these features may be added later on, possibly
as third-party contributions. 

\item Renew is an academic tool. Support will be given as
time permits, but you must be aware that it might take
some time for us to process bug reports and even more time
to process feature requests.

But since Renew is provided with source code, you can do many
changes on your own. And your feature requests have
a high probability to be satisfied if you
can already provide an implementation.
\end{itemize}


\section{How to Read This Manual}

It is generally recommended to read all chapters in
the order in which they are presented. However, when somebody else
has installed Renew for you, you should skip 
Chapter~\ref{ch:install} entirely.

%\newinversion{\renewversion}{%
\newtwodotfive{%
If you are already familiar with a previous version of Renew, you should
simply skim the manual and look for the Renew 2.5 icons as
shown to the left. The paragraphs that are tagged with this
icon elaborate on new features of the current version.

You should also consult Section~\ref{sec:upgrade} for
some notes on the upgrade process. The upgrade might
require some explicit actions on your part.
}

Advanced users may want to consult the architecture guide
\texttt{doc/architecture.pdf} in the source package of Renew,
if it is intended to modify Renew. It is not recommended for the casual
user to spend much time reading this manual, as it is quite technical
and of little help in the day-to-day use of Renew.

\section{Acknowledgements}
We would like to thank Prof.\ Dr.\ R\"udiger Valk and Dr.\ Daniel Moldt
from the University of Hamburg for interesting discussions, help, 
and encouraging comments. 

We would also like to thank 
S\"onke R\"olke,
Dennis Schmitz and
Martin Wincierz
for their work during the preparation of this
release.
We would like to thank J\"orn Schumacher
for the prototype of the plug-in system (2.0), Benjamin Schleinzer
for his work during the preparation of former releases (2.1-2.2) and 
Berndt M\"uller who %%Farwer 
has been of great help with respect to previous
Renew releases for Mac OS ($\leq{}$2.0).
Some nice extensions of Renew were suggested or programmed
by Michael K\"oh\-ler and Heiko R\"olke.

We are indebted to the authors of various freeware libraries, namely
Mark Donszelmann, %%% freehep vectorgraphics
Erich Gamma, %%% JHotDraw
Doug Lea, %%% collections package (still used in FS)
David Megginson, %%% SAX
Bill McKeeman %%% Java Grammar
and Sriram Sankar. %%% JavaCC
%Marc Prud'hommeaux (JLine1), Jason Dillon, Guillaume Nodet, Colin Jones (JLine2)

Dr.\ Maryam Purvis, Dr.\ Da Deng, and Selena Lemalu 
from the Department of Information Science
(\texttt{http://infosci.otago.ac.nz/}), University of
Otago, Dunedin, New Zealand,
kindly aided us in the translation of parts of the documentation
and are involved in an interesting application project.

Valuable contributions and suggestions were made by
students and scientific workers at the University of Hamburg, most notably
Hannes Ahrens,
Tobias Betz, 
Jan Bolte,
Lars Braubach,
Timo Carl,
Dominic Dibbern,
Friedrich Delgado Friedrichs,
Matthias Ernst, 
Max Friedrich,
Daniel Friehe,
Olaf Gro\ss ler, 
Julia Hagemeister,
Sven Heitsch,
Marcin Hewelt,
Jan Hicken,
Thomas Jacob, 
Andreas Kanzlers,
Lutz Kirsten,
Michael K\"oh\-ler, 
Till Kothe,
Annette Laue, 
Matthias Liedtke, 
Marcel Martens, 
Klaus Mit\-rei\-ter,
Konstantin M\"ollers,
Eva M\"uller,
Jens Nor\-gall,
Sven Offermann,
Felix Ortmann,
Martin Pfeiffer, 
Alexander Pokahr,
Tobias Rathjen,
Christian R\"oder,
Heiko R\"olke, 
Benjamin Schleinzer,
Jan Schl\"uter,
Marc Sch\"on\-berg,
J\"orn Schumacher,
Michael Simon,
Fabian Sobanski, 
Volker Tell,
Benjamin Teuber,
Thomas Wagner,
Matthias Wester-Ebbinghaus, 
and Eberhard Wolff. 



We would like to thank the numerous users of Renew who provided
hints and constructive criticism. They helped greatly in improving the
quality of the code and the documentation. In particular, 
we would like to name Alun Champion and Zacharias Tsiatsoulis.

%%% Local Variables: 
%%% mode: latex
%%% TeX-master: "renew.tex"
%%% End: 


%%% Local Variables: 
%%% mode: latex
%%% TeX-master: "renew.tex"
%%% End: 

% LocalWords:  formalisms Petri
