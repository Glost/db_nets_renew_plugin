\chapter{Installation}
\label{ch:install}

In this chapter we will give a short overview of the installation
process. It is not difficult especially if you are already at
ease with the Java environment. But even as a novice you should
be able to complete the process successfully.


\section{Prerequisites}\label{sec:prereq}

Before you proceed, make sure to have a system backup available
on the off-chance that an error occurs during the installation 
procedure.

You must have Java 1.7 or higher installed.
If you have not done this yet, we
suggest that you get the latest Java Runtime Environment from 
Oracle (\url{http://www.oracle.com/technetwork/java/})
where versions for Windows, Linux, Mac OS and Solaris are available. 
%%If you use
%%OS/2, have a look at some 3rd party Java implementations at
%%\url{http://www.goldencode.com/company/software/} or
%%\url{http://www.innotek.de/products/javaos2/}
%For Apple Macintosh you can use the Java runtime build into
%MacOS X, as described on
%\url{http://www.apple.com/}
%without any additional installation.
%You need at least MacOS X 10.5.
All runtime environments are available free of charge for
personal use.
We recommend the latest version of JDK 1.7, which at this point of time 
should be JDK 1.7.0, depending on the operating system.

If you intend to do a source installation, you also need 
to install a couple of software packages from third parties.
See Section~\ref{subsec:sourceinst} for details.


\section{Possible Collisions}

While Renew is based on the 
%% \texttt{collections} package by Doug Lea \cite{Lea98} and the
\texttt{JHotDraw} package by Gamma \cite{Gamma98},
the package is distributed 
with Renew. 
%% In the case of the \texttt{collections} package, this is done for
%% convenience, but the \texttt{JHotDraw} 
The package has been
substantially improved, so that it is impossible to substitute
a different version for it. If you have the original \texttt{JHotDraw}
installed, this might result in a problem.


\section{Upgrade Notes}\label{sec:upgrade}

These notes are supposed to help you when you have already
installed an earlier version of Renew. In Section~\ref{sec:history}
you can find a list of differences, if you are interested in
further details.

\subsection{General}

Note that you cannot usually read nets created with a later
version by older versions of Renew. However, newer versions of
Renew can read older files without problems.
I.e., an upgrade to the current
version is simple, but irreversible. 


\subsection{Upgrade from Renew 1.5 or earlier}

Serialized shadow net systems exported by Renew~1.5 or earlier
cannot be used with any later versions. You can simply re-export
the net system.

A new keyword \texttt{manual} was added to the inscription
language. In rare cases, you will have to rename a variable
or package to account for this change.

\subsection{Upgrade from Renew 1.6 or earlier}

The required Java version has changed, you need at least
Java~1.4 to run newer versions of Renew.

You do not any longer need to install a separate XML parser (like
Xerces) because Renew now uses the built-in parser of Java~1.4.

Saved simulation states exported by Renew~1.6 cannot be used with 
the current version of Renew.

Stubs compiled with Renew~1.6 or earlier cannot be used with 
the current version. You can simply recompile the stubs and the
resulting Java files.

Several classes of the Renew framework were moved or renamed.
In particular, the class \texttt{de.renew.simulator.NetInstance} has now
become \texttt{de.renew.net.NetInstance}. If you use typed
variables for net references in your nets, you must adapt
the variable declarations or package imports.

When a new net is created, the \texttt{:new()} is no longer
implicitly invoked. It is only invoked when you create
the net using the notation \texttt{n:new NetName()} as opposed 
to \texttt{n:new NetName}. You might have to rewrite some
nets as a result of this change.

The way of including custom classes in a Renew simulation has changed,
please read Section~\ref{subsec:classLoading} for more details.

\subsection{Upgrade from Renew 2.0/2.0.1 or earlier}

The main application class has been changed from
\texttt{de.renew.plugin.PluginManager} to \texttt{de.renew.plugin.Loader}.
There are now two configuration file locations: 
first, the per-installation configuration file
\texttt{renew\renewversion{}/config/renew.properties} is read, then
the user-specific file \texttt{\~{}/.renew.properties} is consulted.
The property \texttt{pluginLocations} has been changed to a classpath-like
syntax (see Section~\ref{subsec:installPlugins}).

The net loader priorities have changed: 
instead of preferring \texttt{.sns} files over \texttt{.rnw} files
regardless of the order of their directories in the
\texttt{de.renew.netPath} property, now the order of directories is
considered first, while the order of file types is undetermined.

The Renew code has been modified to compile without warnings under
Java~1.5, but we still use Java~1.4.2.
The conformity with Java~1.5 also implies the use of an up-to-date version
of JavaCC during compilation.
The old \texttt{collections} package has been removed from all components
except FS.
The package is no longer distributed with the base archive, but included in
the FS plug-in.

\subsection{Upgrade from Renew 2.1/2.1.1 or earlier}

The required Java version has changed, you need at least
Java~1.5 to run newer versions of Renew.

Saved simulation states exported by Renew~2.1.1 or earlier cannot be used
with the current version of Renew.
The Drawing Load Server (see \ref{subsec:drawingloadserver} now accepts
connections from the local loopback interface only (this has been
introduced as a security measure).

The JUnit test classes that were sparsely scattered in the Renew code have
been migrated from JUnit 3.x to JUnit 4.x architecture.

\subsection{Upgrade from Renew 2.2 or earlier}

The required Java version has changed, you need at least
Java~6 to run newer versions of Renew.

\subsection{Upgrade from Renew 2.3 or earlier}

Files exported to the experimental \texttt{.xrn} format cannot be used
anymore. Its support has been discontinued. We encourage the use of
PNML, instead.

\subsection{Upgrade from Renew 2.4.3 or earlier}

The logging configuration files and the default logging directory moved from the home folder to a
subdirectory \texttt{.renew} in the home folder. In order to use your
old configuration, you should move the \texttt{.log4j.properties} file
to the \texttt{.renew} subdirectory in your home folder (you may need
to create the directory first). 
For the logs you now have to look in \texttt{.renew/renewlogs} in your home folder (if not configured differently).


\section{Installing Renew}
\label{sec:install}

The two \texttt{zip}-files \texttt{renew\renewversion base.zip} and
\texttt{renew\renewversion source.zip} form the standard 
Renew distribution. While the former file contains all
files that are required for the operation of Renew, the latter file
includes the source files, which are generally \emph{not} needed 
unless you intend to modify Renew or learn about its algorithms. 
In addition to the base distribution, we provide some plug-ins
at our web page.

The base Renew distribution consumes about 5~MByte. 
The source distribution consumes about 18~MByte.
Compiling the source will additionally generate 5~MByte of executables
for the main distribution and the compilation of all available plug-ins
will consume another 30~MByte for additional libraries and
executables.  
The full JavaDoc will occupy between 45~and 57~MByte.

%If you use an Apple Macintosh, refer to Subsection~\ref{subsec:applemac},
%where we provide special installation notes.

\subsection{Base Installation}

In the following, we assume Unix filename conventions, i.e.,
directories separated by \texttt{/}~(slash). For other
operating systems you might need to change it to
\verb:\:~(backslash).
Also, the list separation character differs:
In Unix-based environments, \texttt{:} is used, while in
DOS-based environments, the \texttt{:} is reserved for drive
letters, so \texttt{;} is used for lists.

Extract the base distribution to a directory of your choice.
A directory \texttt{renew\renewversion} will be created
in the current directory. Documentation files, 
for example this manual,
are placed in the subdirectory 
\texttt{renew\renewversion/doc}. The subdirectory 
\texttt{renew\renewversion/samples} contains
example nets. 
The functionality of Renew has been decomposed into several
plug-ins which reside as \texttt{jar} files in the subdirectory
\texttt{renew\renewversion/plugins}. This is also the place where
you can add extra plug-ins (see Section~\ref{sec:plugins}).

The file 
\texttt{renew\renewversion/loader.jar} is a
\texttt{jar}-file that could be used to execute Renew
e.g., by saying
\begin{lstlisting}[style=xnonfloating]
  java -jar /some/where/renew§\renewversion§/loader.jar gui
\end{lstlisting}
if you extracted the \texttt{zip}-file into \texttt{/some/where}.
In a DOS-based environment, this would look something like
\begin{lstlisting}[style=xnonfloating]
  java -jar C:\some\where\renew§\renewversion§\loader.jar gui
\end{lstlisting}
(mind the drive letter and the use of backslash instead of slash).
The \texttt{gui} part at the end of the line is a command that
tells the loader to start the graphical user interface of Renew.

Note that for
Unix, and Windows
we provide ready-made startup scripts already.
They will be generated when Renew is installed. 
In that case, you do not have to provide a loader command
manually, and some classpath-related issues (see
Section~\ref{subsec:classLoading}) are handled more
conveniently for most use cases.
See Section~\ref{sec:plathints} for details.


\subsection{Source Installation}\label{subsec:sourceinst}

Usually there is no need to do a source installation.
If you feel confused by this section, simply skip it.

Extracting the source distribution will put files into the directories
\texttt{renew\renewversion/src} and \texttt{renew\renewversion/doc}.

The sources are accompanied by \texttt{build.xml}
files for the Apache Ant tool.
You should install Ant to build Renew from sources, it is available at
\url{http://ant.apache.org/}.
Further you will need the parser generator JavaCC to compile the sources.
The unit testing package JUnit and the code beautifier Jalopy are used for the development of Renew but optional for the compilation.

See the file \texttt{renew\renewversion/src/README} that lists the
software packages and versions you need and gives some hints on how
to compile and run the application.

After you have successfully built Renew, you can test your compilation
result.
Just follow the instructions in the previous section, but replace any
reference to
\texttt{renew\renewversion/loader.jar} by 
\texttt{renew\renewversion/src/dist/loader.jar}.
The plugins directory is relocated from
its distribution position to
\texttt{renew\renewversion/src/dist/plugins}, too.

If you succeed to run the compiled Renew, you can delete the file
\texttt{renew\renewversion/loader.jar} and the directory
\texttt{renew\renewversion/plugins} with
the original \texttt{class}-files.
You should then consider to edit the start scripts for your
platform, so that they become aware of the file location changes.

Please note that the \texttt{clean} target defined in the Ant
\texttt{build.xml} file completely removes the
\texttt{renew\renewversion/src/dist} directory tree.
Therefore any modifications (like installed plug-ins) made in
that directory tree get lost every time you run \texttt{ant clean}.

\section{Platform-specific Hints}\label{sec:plathints}

For a few platforms we provide special installation support.
Even in these cases you could install Renew as described above,
but your task will be easier if you read this section.

The installation script is typically called
\texttt{installrenew} or similar. Start this script to install
Renew. The script will create the actual startup scripts
for Renew. You have to start the one called \texttt{renew} or similar
to get the basic version of Renew running. 
Other scripts allow you to load drawings into a running Renew editor or
provide help for the generation and compilation of stubs, as summarized in
Table~\ref{tab:startupmodes}.

In the next subsections we will only describe the usage of the
basic script.
The other scripts have a similar behavior when it comes to the
interpretation of environment variables, while their specific effects are
explained in other Sections of this manual.

\begin{table}[h!]
  \begin{center}
    \begin{tabular}{l|l}
      script name & use \\
      \hline
      \texttt{renew} & starts the Renew editor \\
      \hline
      \texttt{loadrenew} & Java drawing load client \\
                         & (see Subsection~\ref{subsec:drawingloadserver}
                         for details) \\
      \hline
      \texttt{makestub} & generates a net stub from a Java interface \\
      \texttt{compilestub} & compiles a net stub to Java code \\
      \texttt{jcompile} & wrapper for \texttt{javac} that includes some \\
                        & Renew classes in the \texttt{CLASSPATH}\\
                        & (see Subsection~\ref{subsec:netmethods} for
                        details) \\
    \end{tabular}
  \end{center}
  \caption{The startup and stub scripts.}\label{tab:startupmodes}
\end{table}


\subsection{Mac OS X}\label{subsec:applemac}

From version~2.4.3 onward, Renew requires at least Java~1.7.

Since Mac OS X is built upon a Unix core, 
you can follow the installation directions for Unix.
To do so, open the Terminal application which can be found in
\path{/Applications/Utilities}.

We provide an experimental AppleUI
plug-in as optional download.
This plug-in adds rudimentary support for native Mac OS look\&feel to
Renew.
It allows the \texttt{cmd-q} (or \texttt{apple-q}) shortcut to close the
editor properly (with safety questions for unsaved changes) and it
slightly modifies the Mac OS menu bar.
See Section~\ref{subsec:installPlugins} on how to install the plug-in.

We also provide an experimental Mac OS X
application bundle for download which can replace the Unix-style
installation of Renew.
The application is configured in GUI-only mode without console
output or prompt capabilities.
Since release 2.2, it also provides Finder integration for Renew
drawing files.

Shortcuts listed in this manual with the \texttt{Ctrl} modifier key should be
used on computers running Mac OS X with the \texttt{Cmd} modifier key
instead .
Some reserved shortcuts differ slightly under Mac OS X (see Appendix~\ref{sec:keyboard-shortcuts}).

\subsection{Unix}

We supply a simple install script
at \texttt{renew\renewversion/bin/unix/installrenew}
that will handle the installation on most flavors of Unix.
Run that script with
\begin{lstlisting}[style=xnonfloating]
  cd renew§\renewversion§/bin/unix
  sh installrenew
\end{lstlisting}
and it will create the shell scripts
\texttt{renew}, \texttt{loadrenew}, \texttt{compilestub}, \texttt{makestub},
and \texttt{jcompile}
in the same directory (see Table~\ref{tab:startupmodes}).

However, you must make sure that \texttt{java} can be
called with your current setting of the \texttt{PATH}
environment variable. It is also required that you start
the installation script from the \texttt{bin/unix} directory,
otherwise it cannot find the location of the package.

We distribute some files to support desktop integration in
FreeDesktop-based environments like Gnome.
In the directory \texttt{renew\renewversion/bin/unix/freedesktop},
there is a \texttt{README} file that explains how such desktop integration
can be achieved.
However, desktop integration still requires a manual configuration
process. The \texttt{installdesktop} and \texttt{uninstalldesktop} scripts automatically install and uninstall the desktop integration in the FreeDesktop environment.

We also provide experimental deb packages for Debian-based
distributions. Just add the following lines to your
\texttt{/etc/apt/sources.list}:
\begin{lstlisting}[style=xnonfloating]
## The Renew Repository
deb http://www.informatik.uni-hamburg.de/TGI/renew/debian renew stable
\end{lstlisting}
  
Download the public key for the Renew repository with:
\begin{lstlisting}[style=xnonfloating]
  wget http://www.informatik.uni-hamburg.de/TGI/renew/renew.gpg
\end{lstlisting}
  
add the key to the trusted keys and install Renew by running the following commands as superuser:
\begin{lstlisting}[style=xnonfloating]
  apt-key add renew.gpg
  apt-get update
  apt-get install renew
\end{lstlisting}


\subsection{Windows}

For Windows we provide an installation script
in the directory \texttt{renew\renewversion}\verb|\bin\win|
for your convenience.
This batch file has to be started from its own directory.
\begin{lstlisting}[style=xnonfloating]
  cd renew§\renewversion§\bin\win
  installrenew
\end{lstlisting}
This will create the batch files \texttt{renew.bat},
\texttt{loadrenew.bat}, \texttt{makestub.bat},
\texttt{compilestub.bat} and \texttt{jcompile.bat}
in the same directory (see Table~\ref{tab:startupmodes}).

However, you must make sure that \texttt{java} can be
called with your current setting of the \texttt{PATH}
environment variable. It is also required that you start
the installation script from the \verb|bin\win| directory,
otherwise it cannot find the location of the package.

Please check your classpath variable for any unquoted whitespace 
in it because batch scripts will interpret paths with it as two 
arguments.

The \texttt{installrenew} script additionally creates some registry files for the file type association of Renew files (\texttt{addregistry.reg} and 
\texttt{removeregistry.reg}).
\texttt{addregistry.reg} contains the registry entries to associate
Renew file types (\texttt{.rnw}, \texttt{.draw}, \texttt{.aip}, \texttt{.sns}) with the \texttt{loadrenew.bat} script.
You can perform this association by double clicking \texttt{addregistry.reg}.
With \texttt{removeregistry.reg} you can remove the corresponding entries from the 
registry.  Admin rights are required to edit registry 
entries. The update of the icons for the Renew file types may need a restart to take effect.


\section{Special Configuration Options}
\label{subsec:options}

There are several options that can help to adapt Renew to
your specific needs. Usually you should not need
to use these options, so it is best to skip this
section on the first reading.


\subsection{Ways of configuring Renew}
\label{subsec:configMethods}

There are at least two ways to configure any of the
options mentioned in the following sections.
In Java terminology, an option is configured by setting a property
to a value.

\paragraph{At startup.}
You can supply property values on the Java command line with the \texttt{-D}
option when you start Renew (this is the way you could configure
previous Renew releases). 
For example, to configure that Renew uses a sequential simulator
instead of the default concurrent one, you can set the property
\texttt{de.renew.simulatorMode} to the value \texttt{-1}.
This is done by starting Renew with the command line:
\begin{lstlisting}[style=xnonfloating]
  java -Dde.renew.simulatorMode=-1 -jar loader.jar gui
\end{lstlisting}
It is important that you do not insert any spaces between the
\texttt{-D} option, the property, the \texttt{=} sign and the value.
Also, property names are case sensitive.

This way, you can configure any property for just one run of Renew. 
However, you lose the simplicity provided by the platform-dependent
startup scripts.
You will have to enter the Java command line directly, including the
full path to \texttt{loader.jar} and the initial command \texttt{gui}
for the loader.

\paragraph{Permanently.}
To configure properties permanently, you can create or edit either the file
\texttt{renew.properties} in the installation directory
\texttt{renew\renewversion/config} or the \texttt{.renew.properties} file (note the initial dot!) in your home
directory.
The former configures the Renew installation, while
the latter provides configuration on a per-user basis.
The user settings may override installation-wide values.
In the following, the text \texttt{.renew.properties} refers to both
files. A template for a \texttt{.renew.properties} file can be found
in the \texttt{config} directory of your Renew installation. The file
contains several out-commented properties. %Furthermore, it contains
%key bindings for English and German keyboard layouts.

The location of your home directory depends on the operating system
(for Windows, it even differs between versions).
If you do not know where it is, just run Renew.
The first two lines of the application log (that is usually printed
to the Java console) tell you where Renew
expects the configuration files.

One note for Windows users: The Explorer will not allow you to
create a file name with an initial dot. But you can create such a
file via the command line interface (DOS window) and then edit it
as usual.

In the \texttt{.renew.properties} file, you can for example add the
line
\begin{lstlisting}[style=xnonfloating]
de.renew.simulatorMode=-1
\end{lstlisting}
and Renew uses the sequential simulation engine by default on every 
run.
Of course, values taken from the \texttt{.renew.properties} file can
always be overridden by a \texttt{-D} command line option at startup.

\paragraph{At runtime.}
A third way for some (not all) of the configuration options is
provided by menus or dialogues in the net editor.
To stick with our example, the dialog opened by the menu entry
\texttt{Simulation}/\texttt{Configure Simulation\dots{}}
contains a checkbox named \texttt{Sequential mode} and a box where you
can change the engine multiplicity (for details, see
Section~\ref{subsec:configureSimulation}).
These two elements in combination represent the current value of the
property \texttt{de.renew.simulatorMode} and its value is changed
when you press the \texttt{Apply} or \texttt{OK} buttons.

As a fourth way to set properties, the plug-in system provides a
\texttt{set} command.
This command comes in useful in combination with the \texttt{script}
command (see Section~\ref{sec:plugins}) or if you installed one of the
prompt plug-ins (Console or GUIPrompt).
Although this command allows to set values for any known and unknown
property at runtime, it is not guaranteed that the value change has any effect.
In fact, many properties are evaluated only once at startup time.

As a rule of thumb, all options that affect the simulation behavior
are evaluated at each simulation setup.
For example, the command
\begin{lstlisting}[style=xnonfloating]
set de.renew.simulatorMode=-1
\end{lstlisting}
has no effect on a simulation currently active (if there is any).
But the next time you start a simulation from within the running
application, it will be sequential.

When you configure properties at runtime, any setting of that property
from the command line or the configuration file is overridden.
Runtime settings are not stored permanently.


\subsection{Drawing Load Server}
\label{subsec:drawingloadserver}

Many users like to load documents into the corresponding
application just by double-clicking the document in
the file manager. A typical problem is that this starts
a new instance of the whole application. This is extremely
nasty when using a Java application, since a new Java Virtual
Machine is started every time.
To avoid this problem with Renew, we added a simple server
function to Renew (the Drawing Load Server). 
This server function is automatically configured to use with the \texttt{loadrenew} script. You probably do not need to do the manual configuration if you just want to pass files into a running Renew instance.


%You probably do not need to do the following configuration by hand. The \texttt{loadrenew} script, generated by the installation script (see Section~\ref{}), uses a preconfigured Drawing Load Server
%
%The loadrenew scripts (see 
%
%The Drawing Load Server is automatically configured 
%
%You may omit the following 


\subsubsection{loadrenew script}

You can use the script \texttt{loadrenew} provided
by the installation process (see Section~\ref{sec:install}) to connect to a running Renew instance and pass over the names of the files to open. The \texttt{loadrenew} script starts a regular Renew instance,
if the connection to a Drawing Load Server was not possible (e.g. if Renew is not already running). So this script can 
be used as default command to load Renew.
The script uses the port configured in the \texttt{.renew.properties}
file (property \texttt{de.renew.loadServerPort}) or the default port 65111 if the property is not set.

The idea is to associate the \texttt{loadrenew} script to the Renew file extensions (dependent on the installed plug-ins but at least \texttt{rnw}).
When a drawing is double-clicked in the file manager,
the client is invoked and can transmit the drawing's file
name, which is then received and loaded by the single
Renew application.
This is really a nice feature, because it offers
a rudimentary operation system integration.


\subsubsection{manual configuration}


By setting the property \texttt{de.renew.loadServerPort} to a port
number, you can tell Renew to listen for local requests to load
a certain drawing on a TCP/IP socket. Just specify an unused
TCP/IP port number, say 65111, either on the command line or in the
configuration file (see Section~\ref{subsec:configMethods}).
When Renew is started with this property,
the ``Drawing Load Server'' is set up on the given
port and waits for clients to connect and transmit the
full filename of a drawing to open (followed by an
end-of-line).
Changing this property's value at runtime has no effect.
The port 65111 is used by default. If you do not want to start
a ``Drawing Load Server',' you can set the property to \texttt{-1}.

We provide a generic client application written in
Java that can be used as a Drawing Load Client. In the meantime,
Java Virtual Machines and computers have become much faster, but it
is still questionable whether it is a good idea to start a JVM each
time you want to load a drawing into Renew.
%To conclude, we still encourage some brave people to write system
%specific Drawing Load Clients (e.g. in C) and make them available
%for the next Renew release!!
%All you need is such a simple client application that
%opens a socket to the server (using the same port number
%as above), transmits its parameter, and closes the
%socket again.

Our Java client is started by
(this example is again given in Unix syntax, the \verb:\: is used to
indicate that the three printed lines should be combined into one
command line)
\begin{lstlisting}[style=xnonfloating]
  java -Dde.renew.loadServerPort=65111 -cp "*:plugins/*:libs/log4j/*" CH.ifa.draw.application.DrawingLoadClient <file name>
\end{lstlisting}

where 65111 is an example for the port number on which the server
is running 
(if you decided to configure the property in the
\texttt{.renew.properties} configuration file, the \texttt{-D} part of
the command can be omitted).




% \subsection{Redraw Strategy}\label{subsec:redraw}

% By default, Renew uses a double buffering algorithm to perform
% display updates, in order to reduce flickering of the screen.

% In some cases, this can expose a Java~1.2 bug that leads to bad screen
% output. Therefore, the double buffering should be disabled for
% Java~1.2 and later versions of Java. Usually, this is done automatically
% but you can force this behavior using
% \begin{lstlisting}[style=xnonfloating]
%   java -DCH.ifa.draw.specialUpdate=true ...
% \end{lstlisting}
% if you experience redraw problems under Java~1.1. If this is the case, 
% please let us know.

% If you want to try buffered display updates using Java~1.2, issue the
% command
% \begin{lstlisting}[style=xnonfloating]
%   java -DCH.ifa.draw.specialUpdate=false ...
% \end{lstlisting}
% and in many cases the flickering will reduce without any negative
% effects.


% \subsection{Resizable Main Window}\label{subsec:resizebug}

% If your Renew window looks quite different than in
% Figure~\ref{fig:renewWindow}, you probably encountered
% a Java bug in the \texttt{pack()} method of \texttt{java.awt.Frame}
% present in (most?) MS Windows Java versions.
% Since the Renew window normally is not resizable, this bug
% can be quite annoying.
% But we provide a workaround: On MS Windows systems, the Renew
% window should automatically be resizable, so that you can
% adjust its size yourself.
% If the bug is not detected automatically, you can start Renew with the
% additional command line parameter
% \begin{lstlisting}[style=xnonfloating]
%   java -Dde.renew.windowResizable=true ...
% \end{lstlisting}
% If the bug is detected mistakenly, i.e.,~the window is displayed correctly,
% but is still resizable, use
% \begin{lstlisting}[style=xnonfloating]
%   java -Dde.renew.windowResizable=false ...
% \end{lstlisting}
% Some X-window managers ignore the setting of a window to not be
% resizable, so we can't do anything about this bug there.

% On some X-Window managers it has been observed that the main window
% jumps off the visible screen area after its creation, so that its cannot be
% accessed. This behavior, too, disappears when you use the option
% \begin{lstlisting}[style=xnonfloating]
%   java -Dde.renew.windowResizable=true ...
% \end{lstlisting}
% to make the main window resizable. Regarding the explanation of this Java
% bug, we have no clue.


%\subsection{Size and position of windows}
%If you get annoyed by a menu window that ``jumps'' after resizing
%it, drawing windows that pop up in different places with
%different sizes each time a drawing is opened or loaded,
%or dialogues that don't display their (full) contents, you probably
%encountered a Java bug related to resizing and positioning of
%windows. We encountered it in the combination Java 1.3.0 with
%the twm window manager on Solaris.
%
%We provide the following workaround: When you set the property
%\texttt{de.renew.avoidFrameReshape} to the value \texttt{true},
%all calls that would set size or position of any window will be
%omitted. This avoids the errors mentioned above in most cases,
%but as a tradeoff all windows will initially pop up in the upper
%left corner of the screen, so that the menu gets covered.
%Since you can move the window away by using the mouse, this
%behavior should be less annoying than the original bug.

\subsection{Multiprocessor Mode}
\label{subsec:concsim}

Renew provides support for shared-memory multiprocessors.
Depending on your specific application and hardware, 
this can significantly speed up the
simulation engine. But note that this feature is
still experimental and has not been tested across platforms
due to lack of funds. It should not be used for critical 
applications at the present time. We would be very glad to 
receive experience reports, so that the code can become more 
stable.

%In order to run Renew on a multiprocessor
%hardware you need a Java implementation that supports native 
%threads. Note that there exist Java implementations that support only
%so-called green threads, which cannot exploit hardware
%parallelism. 

You must set the property \texttt{de.renew.simulatorMode}
to the number of concurrent simulation threads you want.
Note that it will usually be detrimental to
performance, if you configure a number of threads that exceeds the number of
physical processors.
This property is evaluated each time a simulation is started.
It can also be configured in the \texttt{Configure Simulation}
dialog (see Section~\ref{subsec:configureSimulation}).

Before using multiple processors, you should probably try
to optimize performance with other means. You should configure Log4j to
discard log events (alternatively you can disable the generation of
trace events for all or most net elements). You should open
only very few net instance windows, so that the graphical
representation of the markings does not need to be updated.



If you configure a negative number for the property
\texttt{de.renew.simulatorMode}, sequential simulators are used
instead of the default concurrent one.
Although Renew will do that, it is kind of weird to configure multiple
concurrent instances of the sequential simulator.
Only the special case of exactly one sequential simulator is of use and
explained in the next section.


\subsection{Sequential Mode}\label{subsec:seqmode}

By setting the property \texttt{de.renew.simulatorMode} to the value \texttt{-1},
you can request a sequential mode where transition
firings are no longer concurrent.
There is usually little reason to do so, but sometimes
concurrently executed transition inscription might
lead to strange results. You can then select the sequential 
mode to check your nets.

Please note that net methods as described in
Section~\ref{sec:netcall} cannot be used in this mode.

As you might have noted, the property used to configure sequential
mode is the same as for multiprocessor mode.
So the configuration and evaluation notes given in the previous
section still apply to this property.

If we restrict the Petri net formalism to purely
sequential behavior, we can add certain Petri net extensions 
that were not suitable for a true concurrency formalism.
Most notably, we can use inhibitor arcs and clear arcs. 
These extensions will be described in Subsections
\ref{subsec:clearArcs}~and~\ref{subsec:inhibArcs}.

The sequential arc types can now be added to and removed from the
tool bar via the menu entry
\texttt{Simulation}/\texttt{Show sequential-only arcs}
(see Section~\ref{subsec:sequentialTools}).

If you try to simulate nets that contain sequential-only features
and the simulation is configured to use a concurrent engine, you
will encounter an error message.
In this case, you just need to set the \texttt{Sequential mode}
in the \texttt{Configure Simulation} dialog (see
Section~\ref{subsec:configureSimulation}) and restart the
simulation.

%% commented until final release of Maria mode
%\subsection{Maria Mode}\label{subsec:mariamode}
%
%\newTODO{Renew provides an 
%experimental export facility to the Petri net
%analyzer Maria \cite{Maria00}. Because Maria needs to distinguish certain
%net inscriptions graphically instead of syntactically,
%additional functionality was added to the graphical user interface.}
%
%The Maria mode is invoked using
%\begin{lstlisting}[style=xnonfloating]
%  java -Dde.renew.mode=de.renew.gui.maria.MariaMode ...
%\end{lstlisting}
%This mode does \emph{not} allow any simulation of the
%created nets. It is only useful for graphically creating
%Maria nets and exporting them in Maria's own file format
%as described in Subsection \ref{subsec:mariaexp}.

\subsection{Class Loading (and Reloading)}
\label{subsec:classLoading}\label{subsec:classReinit}

When you are developing an application that consists
of Java code and net drawings, or if you want to extend some Renew
plug-in by supplying custom Java classes, then you need to make your
classes known to Renew.
The Java way of doing this is to add your classes to the Java
classpath either by setting the environment variable
\texttt{CLASSPATH} or by providing an appropriate \texttt{-classpath}
option to the \texttt{java} command.

With the plug-in system (see
Section~\ref{sec:plugins}), providing custom classes via the
Java classpath option or variable has some issues.
First, our suggested startup command line uses the \texttt{-jar}
option, which simply overrides any supplied classpath (see the Java
tool documentation for details).
Second, the plug-in system uses its own class loader to find plug-ins.
Due to the Java class loader concept, the plug-in class loader hides
all plug-in classes from any class that is loaded via the system
classpath.
This means that Renew can load and use your custom classes from the
classpath, but your classes cannot refer to any Renew class.
This problem especially affects net stubs -- they need to refer
the \texttt{NetInstance} class of the simulator plug-in.
To work around this, the plug-in system provides the property
\texttt{de.renew.classPath}.

You can set the property \texttt{de.renew.classPath} to any value that
follows the syntax and semantics of the Java classpath (which depends
on the operating system you use).
The property is evaluated once at startup time to configure the class
loader of the plug-in system (changes at runtime are not recommended by
Java's class loader concept).
All classes available through \texttt{de.renew.classPath} can be used
in simulations and as plug-in extensions.
However, if a class with the same qualified name exists in a plug-in
and in the \texttt{de.renew.classPath}, the plug-in class takes
precedence.

For your convenience, the \texttt{renew} and \texttt{loadrenew} startup scripts automatically
transfers the value of the \texttt{CLASSPATH} environment variable to
the \texttt{de.renew.classPath} property at startup.
As long as you stick to these scripts, you should be able
to stick to the classical method of configuring the \texttt{CLASSPATH}
variable to include your custom classes and stubs.

When you are developing an application that consists
of Java code and net drawings, you might want to modify and
recompile your own classes and use them in your nets without
restarting Renew. 
Therefore, we provide a class reloading mechanism to simulations
in Renew.

By setting the property \texttt{de.renew.classReinit} to
\texttt{true},
you can request that all user classes that are referenced by
a net are reloaded before every compilation of a net.
When the class reloading feature is enabled, the
\texttt{de.renew.classPath} property is re-read at each
simulation setup, but changes to the classpath do not affect the
plug-in system.

However, there is a nasty caveat with this feature:
Even without changing the classpath, you can have two instances of the
same class in the Java VM: 
Both are loaded from \texttt{de.renew.classPath}, but one is known
to the plug-in system while the other one is known to the current
simulation only. 
These classes are never identical or equal, if they are compared (for
example when the \texttt{instanceof} operator is applied to an object
of the other class)!
Note that this mechanism may result in some 
problems when you access the Java reflection API, too.

This property can also be changed from the \texttt{Configure
Simulation} dialog (see Section~\ref{subsec:configureSimulation}).
It is evaluated each time a simulation is set up.
If no simulation is running, the graphical editor evaluates it before
each compilation.

Because the reloading of classes might affect performance,
it is disabled by default.


\subsection{Net Loading}\label{subsec:netLoading}

When you are using many nets referencing each other,
you might want to try the net loading mechanism. When nets
are missing during a syntax check, compilation or simulation,
the engine will look for matching drawing (\texttt{.rnw}) or
shadow net (\texttt{.sns}) files and automatically load,
compile and include them into the net system.

The net loader is configured using the property 
\texttt{de.renew.netPath},
where the value is a list of directories specified in
the same syntax as the \texttt{CLASSPATH} environment variable
used by Java. Currently, the net path can comprise directories 
only, \texttt{.jar} or \texttt{.zip} files are not
supported. Subdirectories of the given directories are not
searched, either. The order of directories in the path list
matters, the first directory containing a net will be used.

It is possible to specify netpath directories
relative to the classpath.
This is done by prepending the directory entry with the (reserved)
directory name \texttt{CLASSPATH}.
For example, if you include the directory (in Unix syntax, Windows
users should replace the slash by a backslash)
\texttt{CLASSPATH/nets}
in the \texttt{de.renew.netPath} property, then the 
net loader would look for a missing net file throughout all
\texttt{nets} directories relative to all classpath entries.
The searched classpath includes everything from the Java system
classpath, all loaded plug-ins, and the \texttt{de.renew.classPath}
property (see Section~\ref{subsec:classLoading}), in that order.
When they are included in the classpath, \texttt{.jar} and
\texttt{.zip} files are searched, too.

The behavior of the net loader differs depending on the type of
file it finds when looking for a net. If it encounters a shadow
net system (\texttt{.sns}) file, the net will be loaded into the
simulation or used for a syntax check, but it will not show up in
a drawing window on the screen.
Matching shadow net system files must contain a single net with the same
name alone.
If more than one net is found in the shadow net system file, it is rejected
to avoid confusion about the source of compiled nets.  
There is a command in the menu \texttt{File}/\texttt{Export}/\texttt{Export
  all (single file each)} to generate such single-net shadow net system
files for all open drawings (see Subsection \ref{subsec:expshadow}).

When the simulation has been started via the \texttt{Simulation} menu in
the editor (and \emph{only} then!), the net loader will also look for
(\texttt{.rnw}) files in the \texttt{netPath}.
If it finds a matching drawing file, the drawing shows up in an editor
window and is immediately compiled into the running simulation.

If a net can be obtained from several different sources at the same time,
the net loader takes the first one in the order of the \texttt{netPath}
entries.
If it encounters \texttt{.sns} files in the same directory as
\texttt{.rnw} files with the same name, it is not defined which one gets
loaded.

The editor also uses the net loader during simulation to open drawings
from \texttt{.rnw} files that are needed to display net instances.
But you should be aware that the net loading feature comes with a big
pitfall:
If \texttt{.sns} files and \texttt{.rnw} files are not kept up-to-date,
compiled nets in the running simulation \emph{may} have a different
structure than the nets shown in the editor!
Since the mapping from a simulated net to its drawing is based solely on
the net name, it is possible that transitions and places in a net instance
window do not fire in accordance to the visible net structure.

The \texttt{de.renew.netPath} property is re-read every
time a simulation is started.
It is configurable at runtime in the editor's \texttt{Configure
Simulation} dialog (see Section~\ref{subsec:configureSimulation}).

\subsection{Database Backing}

You can run Renew using a database that keeps a persistent
copy of the current simulation state. 
In the case of a system crash, this allows 
you to continue the simulation from a valid state just before 
the crash when Renew is restarted.
Database backing is only supported when the simulation
is started from the command line (see Section~\ref{sec:simulationServer} for information on controlling a simulation from command line).
Using database backing in gui mode may lead to unexpected behavior
But note that this feature is
still experimental and has not been tested across platforms
due to lack of funds.

The setup of the persistent database backing
is definitely more tricky than the
other Renew options, but it is supported by some predefined 
scripts. The source package contains among others the
SQL script \texttt{initTable.sql} which can be found in
\path{Simulator/src/de/renew/database/entitylayer/}.

It creates the required database tables for an Oracle server. For
other databases, the script will need some changes, but even the
various versions of Oracle differ enough to cause minor problems. A
database backend, which supports transactions is required (e.g.\ MySQL
does not support transactions when using the default MyISAM engine;
use InnoDB instead).

Having created the tables, you should configure a set of properties to
enable the database backing feature (see
Table~\ref{tab:simdbProperties}).
These properties are evaluated each time a simulation is set up.
\begin{table}[htbp]
  \centering
  \begin{tabular}{llp{7cm}}
    Property name                    & Type   & Comment\\
    \hline
    \texttt{de.renew.simdb.driver}   & class  & JDBC driver class (mandatory).\\
    \texttt{de.renew.simdb.url}      & URL    & JDBC connection URL (mandatory).\\
    \texttt{de.renew.simdb.dialect}  & class  & Subclass of \texttt{de.renew.database.entitylayer 
                                                .SQLDialect} (optional).\\
    \texttt{de.renew.simdb.user}     & string & User account for database login (optional).\\
    \texttt{de.renew.simdb.password} & string & Password for database login.
                                                (ignored when \texttt{de.renew.simdb.user} is not set).
  \end{tabular}
  \caption{Properties to configure database backing}
  \label{tab:simdbProperties}
\end{table}
The class names specified for the \texttt{driver} and \texttt{dialect}
properties should be accessible via the \texttt{de.renew.classPath}
(see Section~\ref{subsec:classLoading}).
The \texttt{dialect} class is an internal class that adapts
Renew to the SQL dialect of your database.
The default is the generic \texttt{SQLDialect}, but for some databases
we already provide experimental implementations like
\texttt{OracleDialect}, \texttt{MySqlDialect} or \texttt{MSqlDialect}.

When using the database backing, your nets must conform
to certain restrictions. Unfortunately, these restrictions
cannot be checked by Renew automatically, so that you must take
special care when preparing your net. 

All tokens used in your net must be serializable, i.e.,
custom classes must implement \texttt{java.io.Serializable}.
Typically, all tokens are also immutable \emph{value objects},
which acquire their state once during creation, before these 
objects are used as tokens in the net.
For value objects, the \texttt{equals()} method must not be based on
object identity, but on the represented value. 
Similarly, the \texttt{hashCode()} method
must also be properly defined.

If you use mutable, stateful objects in your nets, you
must observe further restrictions. Contrary to value object,
stateful objects must preserve the original implementations
of \texttt{Object.equals()} and \texttt{Object.hashCode()}.
Furthermore, the stateful objects must either
occur directly as tokens in the net or there must be exactly
one token by which a given stateful object is reachable.
Failure to do so will result in a corrupted simulation
state when restoring the simulation from the database.

There is no garbage collection when using the database-backed
simulation.

\subsection{Remote Simulation Access}
\label{subsec:remoteSetup}

Any Renew simulation, regardless whether it is
started from the command line or within the graphical editor,
can be published via Java's Remote Method Invocation (RMI)
technique.
Any Renew editor can then connect to the published remote
simulation, display the token game, and control the firing of
transitions.
See Section~\ref{subsec:connectToServer} about how to connect to
a running remote simulation.
This section focuses on how to configure the simulation engine
for remote access.

The first step is that you start an RMI registry on the machine where
the server will be running. This is a program distributed together
with Java that stores RMI objects and makes them accessible to other
VMs. Simply run the program \texttt{rmiregistry}
(e.g.\ as a background task).
Note that the \texttt{rmiregistry} process either needs a
classpath with \texttt{renew\renewversion/plugins/\PluginRemote{}}
included or that the simulation server has to be configured with
the \texttt{java.rmi.server.codebase} property (see Java RMI docs
for details).

%rmiregistry -J-Djava.rmi.server.codebase="file:<path/to/>\PluginRemote{}"
%de.renew.remote.enable=true
%de.renew.remote.rmi-host-name=<IPthisComputer>

\begin{table}[htbp]
  \centering
  \begin{tabular}{llp{6cm}}
    Property name                          & Type    & Comment\\
    \hline
    \texttt{de.renew.remote.enable}        & boolean & Enables remote access
                                                       (defaults to \texttt{false}).\\
    \texttt{de.renew.remote.publicName}    & string  & Name to use for RMI registration
                                                       (defaults to \texttt{default}).\\
    \texttt{de.renew.remote.serverClass}   & class   & Implementation of the interface
                                                       \texttt{de.renew.remote.Server}
                                                       (defaults to \texttt{de.renew.remote.ServerImpl}).\\
    \texttt{de.renew.remote.socketFactory} & class   & Implementation of the interface 
                                                       \texttt{java.rmi.server.RMISocketFactory}
                                                       (defaults to RMI's default factory).\\
    \texttt{de.renew.remote.rmi-host-name} & string   & the rmi server hostname to the correct ip if 
the remote server cannot be found. It is similar to the java property 
\texttt{java.rmi.server.hostname}\\                                                     
  \end{tabular}
  \caption{Properties to configure a remote simulation}
  \label{tab:remoteProperties}
\end{table}

The remote simulation can be configured by using the properties
listed in Table~\ref{tab:remoteProperties}.
The defaults are suitable for most cases, so you just need to
enable the remote access by setting the property
\texttt{de.renew.remote.enable} to true.

The \texttt{publicName} property is required, if you intent to
run several servers on one machine. When connecting from an
editor, you can specify the server to connect to by its name.

The \texttt{serverClass} and \texttt{socketFactory} properties
are normally not needed.
Plug-in developers may use these properties to replace the default
implementations by enhanced versions.
The server class determines the simulation control features which
are remotely accessible.
The socket factory may enhance RMI transmissions by compression
and/or encryption of the network traffic.

All properties except \texttt{socketFactory} are re-evaluated at
each simulation setup.
They are configurable at runtime in the editor's \texttt{Configure
Simulation} dialog (see Section~\ref{subsec:configureSimulation}).

\subsection{Logging}
\label{subsec:log4jConfiguration}

Renew uses the Log4j package (in version~1.2.x) from the Apache
Logging Services project \cite{Log4j12} to give detailed feedback about
its activities.
The Log4j framework allows users to tailor the logging output to fit
their needs.
The level of detail can be configured for every application package or
class individually.%

The Renew base installation contains a binary distribution of Log4j in the
\texttt{renew\renewversion{}/libs} directory.  
The package is loaded immediately at application startup, before any
configuration of the plug-in system is done.
The early load time of this package has the consequence that the enhanced
configuration methods mentioned at the beginning of this Section are
\emph{not applicable} to configure the logging system.

\subsubsection{Configuration mechanism}
The configuration options of Log4j would occupy to much space in this
manual, please have a look at the documentation section of the Log4j
homepage \cite{Log4j12}.
With the Renew distribution comes a commented default configuration file
\texttt{renew\renewversion/config/log4j.properties}.

Renew looks for Log4j configuration files at similar locations like its own
\texttt{.renew.properties} files (see Section~\ref{subsec:configMethods}).
The configuration files can be in XML syntax or in the Java properties
format and must carry appropriate file extensions.
The first file found in the following list of candidates is used:
\begin{enumerate}
\item A file named in the system property \texttt{log4j.configuration} at
  the \texttt{java} command line with the \texttt{-D} option.
\item \texttt{.log4j.xml} in the \texttt{.renew} subdirectory of your home directory.
\item \texttt{log4j.xml} in the installation directory \texttt{renew\renewversion/config/}.
\item \texttt{.log4j.properties} in the \texttt{.renew} subdirectory of your home directory and
  \linebreak \texttt{log4j.properties} in the installation directory
  \texttt{renew\renewversion/config/}.  
  If both files exist, individual settings in the user file take precedence
  over settings in the installation file.
\item A failsafe setup with a simple console logger is used if all
  configuration files are missing.
\end{enumerate}

Renew provides the \texttt{logs.home} property that can be referred from
within the configuration files.
This property by default points to the directory \texttt{.renew/renewlogs} in your
home directory, but you can override that setting with the \texttt{-D}
option on the \texttt{java} command line.

\newtwodotfive{%
With Renew 2.5 the user configuration files \texttt{.log4j.properties} and \texttt{.log4j.xml} as well as the \texttt{logs.home} default location moved from the home directory to the \texttt{.renew} subdirectory in your home directory.
}


\subsubsection{Default configuration}
The default configuration file
\texttt{renew\renewversion/config/log4j.properties} uses the
\texttt{logs.home} property (see above) so that you get two log files
(\texttt{renew.log} and \texttt{simulation.log}) in that directory.
The former comprises application messages while the latter stores
simulation traces.
In the default configuration, application log messages are also printed to
the Java console, but the simulation trace is not.
The logging plug-in provides a graphical user interface that displays the
simulation trace and allows flexible configuration of logged events (see
Section~\ref{subsec:loggingPlugin}).


\section{Plug-ins}
\label{sec:plugins}
As of Renew 2.0, the application is controlled by
a plug-in system.
The system is started through the class
\texttt{de.renew.plugin.Loader}.
The loader sets up some class libraries and then loads the main plug-in
manager.
The plug-in manager finds, loads, initializes and terminates
plug-ins, but it knows nothing about Petri nets.
The Renew functionality is provided by a set of plug-ins.
More plug-ins can be installed to extend Renew.

In this section, there will be a lot of examples with file names and
directories.
These are all given in Unix syntax.
Users of other operating systems: please transform these
examples to your appropriate syntax.

\subsection{Install Plug-ins}
\label{subsec:installPlugins}
There are two ways to install a plug-in.
If it comes in one single \texttt{.jar} file which includes a file
named \texttt{plugin.cfg}, you can just copy the file in the
\texttt{renew\renewversion/plugins} directory
(if you use the Mac OS X application bundle, this directory is located
\emph{inside} the application package at the path
\texttt{Contents/Resources/Java/plugins}).
If the plug-in comprises several files, one of which is the file
\texttt{plugin.cfg}, then you can create a subdirectory below
(e.g. \texttt{renew\renewversion/plugins/myplugin} and copy all
files in this directory.

On the next startup, the plug-in system will find and include the
plug-ins automatically.
If you want to include the plug-in in the running system, use the
\texttt{load} command (see Section~\ref{subsec:pluginCommands})
in addition.

If you do not want to install the plug-in to your renew
distribution directory, you can install it to some other
directory, let's say 
\texttt{/home/myself/devel/myplugin.jar} (as single-jar plug-in)
or \texttt{/home/myself/devel/myplugin/} (as multi-file plug-in).
Then you again have two possibilities:
To make the plug-in system aware of the plug-in at the next
startup, add the line
\begin{lstlisting}[style=xnonfloating]
  pluginLocations=/home/myself/devel
\end{lstlisting}
to your \texttt{.renew.properties} file (see
Section~\ref{subsec:configMethods} for details).
When entering multiple paths, they must be separated by the
system's path separator (which is ";" on Windows and ":" on Unix systems).
To load the plug-in immediately but temporarily into a running
plug-in system, use the \texttt{load} command:
Depending on the number of files comprising the plug-in, it's
one of the following commands:
\begin{lstlisting}[style=xnonfloating]
  load file:/home/myself/devel/myplugin.jar
  load file:/home/myself/devel/myplugin/plugin.cfg
\end{lstlisting}

\subsection{Exclude Plug-ins Temporarily}
\label{subsec:excludePlugins}
To hide installed plug-ins from the plug-in finder at startup, you
can specify the property \texttt{de.renew.plugin.noLoad} either
via \texttt{-D} command line option or the
\texttt{.renew.properties} file (see
Section~\ref{subsec:configMethods} for details).
The value of the property is a comma separated list of plug-in
names.
For example, the line
\begin{lstlisting}[style=xnonfloating]
  java -Dde.renew.plugin.noLoad="Renew JHotDraw" -jar renew§\renewversion§/loader.jar gui
\end{lstlisting}
will start the plug-in system, but terminate with the complaint that the
\texttt{gui} command is unknown.
Because the JHotDraw plug-in has not been loaded, all dependent
plug-ins can also not be loaded.
This affects the Renew Gui plug-in which would otherwise have provided
the \texttt{gui} command.
Of course it would make more sense to use some non-graphical
command like \texttt{startsimulation} instead.

Alternatively it is possible to specify only the plug-ins you want to load at startup. For that you have to set the property \texttt{de.renew.plugin.autoLoad} to \texttt{false}. And specify the plug-ins you want to load with the \texttt{de.renew.plugin.load} property. The following line will also start the plug-in system and terminate, because of the missing \texttt{gui} command. 

\begin{lstlisting}[style=xnonfloating]
  java -Dde.renew.plugin.autoLoad=false -Dde.renew.plugin.load="Console,Renew Simulator,Renew Formalism,Renew Util" -jar renew§\renewversion§/loader.jar gui
\end{lstlisting}

\subsection{System Termination}
\label{subsec:systemExit}
The plug-in system tries to detect the situation where no plug-in
is active and therefore the system can be shut down.
Plug-ins are ``active'' if they have some long-term work to do.
This always holds for the windows of the graphical editor.
A running simulation also counts as active.
The Console plug-in has a special keep-alive flag which marks it
as active (see Section~\ref{subsec:consolePlugin}).

Besides automatic termination, the plug-in system can be
terminated by request.
The \texttt{exit} command (see
Section~\ref{subsec:pluginCommands}) has just that purpose.
The editor may also terminate the plug-in system when it's main
window is closed. 
The editor does this not by default, you have to set the property
\texttt{de.renew.gui.shutdownOnClose} to \texttt{true}.

The property \texttt{de.renew.gui.autostart} automatically starts
the editor without the need for an initial \texttt{gui} command,
when set to \texttt{true}.
The combination of these two gui properties frees users of the
pure graphical editor of most complications introduced by the
plug-in system.


\subsection{Commands}
\label{subsec:pluginCommands}

As mentioned in Section~\ref{sec:install}, the plug-in system
needs an initial command to start some plug-in.
Any plug-in can provide such commands (although the \texttt{gui}
command is the one that you will use most of the time). In the
following, we present the basic commands provided by the plug-in
manager itself and some additional commands provided by other
plug-ins. Note that it is also possible to define a chain of commands
by separating the commands with \texttt{---}. Most of the commands
presented here you will typically not use on Renew start-up, but
rather in combination with the Console plug-in described in the
following Section~\ref{subsec:consolePlugin}.

A basic set of commands is provided by the plug-in manager itself:
\begin{description}
\item[\texttt{help}]
  prints a list of all available commands.
  Due to the addition or removal of plug-ins, this list may vary from
  time to time.
\item[\texttt{get}]
  prints the value of a property. The property name has to be given as
  an argument.
  \begin{itemize}
  \item[\texttt{-a}] shows all known property keys.
  \end{itemize}
\item[\texttt{set}]
  sets the value of a property (as explained in
  Section~\ref{subsec:configMethods}).
  This command accepts multiple arguments of the form
  \texttt{key=value}.
  It is important that no spaces are included in the key, the
  value, or in between.
\item[\texttt{list}]
  prints a list with the names of all loaded plug-ins.
  The command respects some mode switches:
  \begin{itemize}
  \item[\texttt{-l}] (or \texttt{--long}) shows date and version
    information of plug-ins (if available).
  \item[\texttt{-j}] (or \texttt{--jar}) shows the jar file locations of
    plug-ins and libraries.
  \item[\texttt{-o}] (or \texttt{--ordered}) shows an alphabetically ordered list of plug-in names.
  \end{itemize}
\item[\texttt{info}]
  prints information about one plug-in.
  The plug-in's name has to be specified as command argument (use
  \texttt{list} to see the plug-in names).
\item[\texttt{load}]
  loads one plug-in dynamically, if possible.
  The argument to this command is a URL specifying the plug-in's
  location.
  The plug-in location can be given relative to the
  \texttt{de.renew.pluginLocation} directories.
  For example, \texttt{load file:gui.jar} would load the gui
  plug-in from the renew distribution
  \texttt{renew\renewversion/plugins/gui.jar}. You can use wildcards (e.g. \texttt{load gu*}), if there is an ambiguity the alternatives are prompted for selection of the correct one.
  
  
\item[\texttt{unload}]
  terminates and unloads a plug-in, if possible.
  The plug-in's name has to be specified as command argument (use
  \texttt{list} to see the plug-in names).
  If other plug-ins depend on the given plug-in, the plug-in system
  will complain.
  You can add the argument \texttt{-v} to see a list of
  dependent plug-ins, or the argument \texttt{-r} to unload all
  recursively dependent plug-ins.
  \bug{Although the plug-in is terminated and all dependencies are
    cleaned, its classes are still accessible.
    Fixing this bug requires a different plug-in class loader,
    which will hopefully be written in some future release.}
\item[\texttt{packageCount}] prints the packages and the total number
  of packages in the class loader.
\item[\texttt{exit}]
  terminates the plug-in system, and in consequence the whole
  application.
  If some plug-in hangs during termination, you can use
  \texttt{exit force} to kill the Java VM abruptly or \texttt{exit ifidle} to exit only if all plug-ins are inactive.
\item[\texttt{gc}] triggers the Java garbage collector.
\item[\texttt{script}]
  loads commands from a text file and executes them.
  The file name has to be given as argument, it can be specified
  relative to the current directory.
  This command is especially useful as initial command, when you
  in fact need to issue several commands at startup of the plug-in
  system.
\item[\texttt{sleep}] This command waits for a given time until the next command is executed. The time to wait is given as argument in seconds.
\end{description}

The following commands are not provided by the plug-in manager,
but by some Renew plug-ins.
So they are available only when the respective plug-in is loaded.
\begin{description}
\item[\texttt{gui}]
  starts the graphical editor and/or passes its arguments to the
  editor. 
  The arguments are supposed to be drawing file names.
  This command is provided by the Gui plug-in.
\item[\texttt{demonstrator}] opens a window with a list of drawing
  file names. This plug-in is provided by the Gui plug-in.
\item[\texttt{ex}] exports a drawing into various formats. The usage
  of the command is \texttt{ex <type> <drawing>} where
  \texttt{<type>} may be (but may vary
  depending on the installed plug-ins):
  \begin{itemize}
  \item \texttt{PNG}
  \item \texttt{PNML-RefNet}
  \item \texttt{PNML-http://www.informatik.hu-berlin.de/top/pntd/ptNetb}
  \item \texttt{SVG}
  \item \texttt{ShadowNetSystem}
  \item \texttt{Woflan}
  \end{itemize} 
  This command is provided by the Export plug-in which
  in turn uses the FreeHEP project for graphical exports. The command respects some mode switches:
    \begin{itemize}
  \item[\texttt{-a}] (or \texttt{--accumulate}) n-to-1 export (only available for some formats, e.g. ShadowNetSystem).  
  \item[\texttt{-o}] (or \texttt{--output}) specify the output file.
  \end{itemize}
  You can type just \texttt{ex} without any options to see the available formats.
\item[\texttt{startsimulation}]
  starts a simulation without using the graphical editor. 
  See Section~\ref{sec:simulationServer}.
  This command is provided by the Simulator plug-in.
\item[\texttt{simulation}]
  controls a running simulation without using the graphical
  editor.
  Use one of the subcommands \texttt{help}, \texttt{run},
  \texttt{step}, \texttt{stop} or \texttt{term} as argument.
  See Section~\ref{sec:simulationServer}.
  This command is provided by the Simulator plug-in.
\item[\texttt{setFormalism}]
  chooses the formalism given as argument as current formalism.
  This is equivalent to selecting a formalism in the menu (see 
  Section~\ref{subsec:formalismgui}).
  This command is provided by the Formalism plug-in.
\item[\texttt{listFormalisms}]
  lists all installed formalisms.
  The names listed by this command can be used as argument to the
  \texttt{setFormalism} command or as value for the
  \texttt{renew.compiler} property.
  This command is provided by the Formalism plug-in.
\item[\texttt{keepalive}]
  displays and manipulates the keep-alive feature of the Console
  plug-in (see Section~\ref{subsec:consolePlugin}).
\item[\texttt{guiprompt}] 
  opens a graphical prompt dialog (see Section~\ref{subsec:consolePlugin}),
  if the Gui Prompt plug-in is installed.
\item[\texttt{navigator}]
 opens the Navigator window.  
\end{description}

\subsection{Console}
\label{subsec:consolePlugin}

The Console plug-in is part of the base distribution.
It enhances the plug-in system by an interactive command shell in
the Java console window and allows to
issue commands to the plug-in system at runtime.

\newtwodotfive{
 With Renew 2.5 the old prompt plug-in is replaced by the new Console plug-in. It utilizes JLine2 as library and provides editable command lines, tab-based command and argument completions as well as searchable and persistent command history. 
}

If you start the renew from a shell, you will be presented
with the prompt \texttt{Renew >} at the Java console.
Here you can enter any of the commands mentioned in the previous
subsection.
The Console plug-in itself offers one command with two alternative
arguments:
\begin{description}
\item[\texttt{keepalive on}]
  enables the keep-alive feature.
  As long as this feature is on, the plug-in system will not
  terminate automatically because there is no active plug-in (see
  Section~\ref{subsec:systemExit}).
  However, an explicit termination request will still be
  executed.
\item[\texttt{keepalive off}]
  turns the keep-alive feature off.
\item[\texttt{keepalive}]
  without argument displays the current state of the keep-alive
  flag.
\end{description}
The initial state of the keep-alive flag is determined from the
value of the property \texttt{de.renew} \texttt{.console.keepalive}.
If you tend to use the Console plug-in as your only active plug-in most of
the time, you should consider adding the line
\texttt{de.renew.console.keepalive=true} to your
\texttt{.renew.properties} configuration file. 

As an alternative to the Console plug-in, there also exists a
gui prompt plug-in for download.
This plug-in adds a small dialog to the \texttt{Plugins} menu of
the editor which accepts commands to the plug-in system.
The graphical prompt automatically keeps the plug-in system alive as long
as its dialog is open.
The dialog can also be opened by the command-line command
\texttt{guiprompt}.
The command feedback is now visible in the dialog instead of the Java
console window.

It is possible to escape whitespace characters by surrounding double quotes 
or with a preceding backslash (e.g. \texttt{gui "foo bar/file.rnw"} or  
\texttt{gui foo\char`\\{} bar/file.rnw}).

\subsection{Net Components}
\label{sec:net-components}

Net Components (NC) are sets of net elements (Figures) that are
grouped in a flat and weak fashion.
%
The aim is to be able to allow to move the whole set of net elements
(the net component) in convenient way.
%
\emph{Weak} means that, although the net component can be moved by
clicking and dragging the mouse in between the net elements, they can
each individually be manipulated. Individual net elements can be
dragged resulting in a manipulated layout of the net component. It is
also possible to edit inscriptions of elements belonging to the net
component.
%
\emph{Flat} means that grouping is not hierarchical.  
%
In order to provide the functionality a new Figure has been added,
which has no graphical representation of its own.

The Net Component Plugin provides two basic functionalists. 
%
The grouping and ungrouping of a selection of net elements and the
management of tool palettes, which represent a repository of (pre-)
defined sets of net components.
%
A repository consists in a folder that holds several Renew net
drawings, an images folder and a configuration file (\emph{.sequence}).
%
The images folder should contain icons in gif format (24$\times$24
pixels), which are used to fill the tool buttons of the palettes.
%
If no image is defined for a specific net component a generic image
(\emph{generic.gif}) will be used instead.
%
Names of net drawings and icon images should correspond.
%
The \emph{.sequence} file contains the names of the drawings (without
extensions) and define the sequence in which the buttons are shown in
the palette.
%
The file may be empty but has to be present in order for the directory
to be recognized as net component repository.
%
Further, non-listed drawings are included in the palette without
specific order.

There are two ways to use the net component repositories within the
plug-in.
%
First, repository folders can be opened directly using the menu command.
%
Second, plug-ins can extend the functionality of the NC plugin by
providing a repository.
%
Such a plugin can use the provided generic plugin facade
\emph{PalettePlugin} or a customized facade class can be provided.
% 
In the latter case also other tool buttons and command line commands
can be added to the system.
%
Plugins using the \texttt{PalettePlugin} class must provide two
properties for the definition of the directory and two for the
automatic initialization switch: \texttt{de.renew.nc.dir-prop-name},
\texttt{de.renew.nc.init-prop-name}.
%
A minimal example for a plugin configuration file (\emph{plugin.cfg})
is presented below.
%
More information about the concept and the tool can be found
in~\cite{Cabac09c}.

\begin{lstlisting}[style=xnonfloating]
    mainClass = de.renew.netcomponents.PalettePlugin
    name = Renew ExampleComponents
    provides = com.example.nc
    requires = de.renew.util,de.renew.gui,de.renew.nc
    de.renew.nc.dir-prop-name=com.example.nc.dir
    de.renew.nc.init-prop-name=com.example.nc.init
    com.example.nc.dir=tools
    com.example.nc.init=false
\end{lstlisting}

\section{Troubleshooting}

A few possible problems and their solutions are described here.
If you have problems, contact us. If you have solutions,
contact us, too.

\begin{itemize}
\item I cannot extract the files from the archives.

Maybe the files got corrupted on their way to you?
Maybe you are using an old version of \texttt{unzip}?
If you have a version of the JDK that does not support zipped
\texttt{jar} archives, please let us know.

Note that you must use Java~1.1 or newer to use \texttt{jar} archives
and that Renew requires Java 1.7 for different reasons anyway.

\item Java is not found.

Probably the shell scripts try to look for Java in the wrong places.

\item Under Windows, the installation or the start of Renew
terminates with a message regarding missing space for environment
variable.

Enlarge the environment space for the command window.

\item Java cannot find the class \texttt{de.renew.plugin.Loader}.

This should not happen if you use \texttt{-jar renew\renewversion/loader.jar}
at the Java command line.
If you want to use the environment variable \texttt{CLASSPATH}
instead, check if it includes \texttt{loader.jar}.

\item Renew starts, but the window titles are incorrect under
the X Windows System.

Try a different window manager, e.g., \texttt{mwm} is known to work
correctly.
This is a general Java problem and not related to Renew, so we cannot do
anything about it.

\item When I switch to a window of a different application while the
Renew menu is open, the menu stays visible in front of all other windows.

This is a general Java (more concrete: Swing) problem and not related
to Renew, so we cannot do anything about it.
With Java 1.5, the bug has been fixed.

% \item My Renew window is too large or too small, so that it consumes too
% much space or (even worse) the tools are not reachable.

% See Subsection~\ref{subsec:resizebug} for a workaround.

% \item The Renew window does not appear or it does appear, but
% jumps off the screen after a short time.

% See Subsection~\ref{subsec:resizebug} for a workaround.
% Alternatively, under Unix you might try a different window manager.

% \item Renew does not properly display the contents of a drawing.
% Renew redraws only part of the screen.

% See Subsection~\ref{subsec:redraw} for a workaround.

%\item When I drag around text figures, Renew displays some spurious
%dots on the screen.

%We encountered this when working with certain Java 1.2
%virtual machines under Solaris. This seems to be a Java bug.
%The only known remedy is to use another virtual machine.

\item I cannot open the sample files.

Sometimes you need to add the root directory~\texttt{/}
(or~\verb:\:, depending on your operating system) to your class path.

\item Under Mac OS X, using \texttt{cmd-Q} (\texttt{apple-Q})
to quit the application will not give you the opportunity
to save changes, even if the document has recently been modified.

The optional AppleUI plug-in (see
  Section~\ref{subsec:applemac}) solves that problem.
\end{itemize}

\section{History}\label{sec:history}

Version 1.0 was the first public release. It
included a net editor, a reference net
simulator, a Java stub compiler, and example nets.

\subsection{Changes in Version 1.1}

\subsubsection*{Modifications}

Some performance enhancements were implemented and minor
bugs were fixed. Some source level inconsistencies were
cleaned up.
The thread model of Java 1.2 was adopted.
The source code was changed to be compilable with Java 1.1.3.
The windowing code was made more robust under Java 1.2.

The handling of \texttt{null}-objects in the simulator was corrected.
The type system was made more compatible with the Java type system.
The trace flag of netelements is now saved to disk.
The simulation performance was improved.
The garbage collection of net instances was improved.

The graphical user interface was improved for some window managers.
The presentation of current markings was improved.
The interactive execution of reference nets has been
improved a lot (see Section~\ref{sec:simulation}).

\subsubsection*{Additions}

The parallel simulation code was added.
The checks for double names and for cyclic channel
dependencies were added.
Transition inscriptions may now include several parts
separated by semicolons.
Virtual places may now be used in nets.

During the simulation, bindings can be selected and fired
under user control.
The multiset of tokens contained in
a place instance can be displayed as just the cardinality of the
multiset, a collection of all tokens in the multiset directly
within the drawing, or in a separate window.
Individual components of tuples can
be inspected. Initial markings are hidden during the simulation.

\subsection{Changes in Version 1.2}

\subsubsection*{Modifications}

The simulation engine was made more robust and flexible.
Minor bugs were fixed.

A single inscription figure may now contain multiple
arc inscriptions or initial marking inscriptions that are separated
by semicolons.
Slight inconsistencies in the inscription language were
cleaned up. The type rules were improved.
The results of action inscriptions may now be passed through
synchronous channels even in the presence of typed variables.

Some display problems with Java 1.2 have been fixed.

\subsubsection*{Additions}

Flexible arcs were added.
Clear arcs were added.
Inhibitor arcs were added.

Marked places and firing transitions can now be highlighted during
the simulation.
A rudimentary net layout algorithm has been implemented.
The state of a running simulation can now be saved and restored.
Restarting a simulation may now reload Java classes.

Export of Encapsulated PostScript was implemented.
Selection of groups of figures was improved.

\subsection{Changes in Version 1.3}

\subsubsection*{Modifications}

Some minor improvements of the graphical user interface were
applied.

\subsubsection*{Additions}

The timed simulation mode was added.
Lists were provided in addition to tuples.

Breakpoints were added in order to control the
graphical simulation.
An XML import and export facility was added.
A graph layout algorithm that may help in viewing nets
was added. More commands for arranging figures manually were
provided. The ability to select and deselect figures by type
was added.

\subsection{Changes in Version 1.4}

\subsubsection*{Modifications}

This was a maintenance release that provided mainly
improvements in the user interface, documentation updates, and
bug fixes.

\subsubsection*{Additions}

You can now inspect token Java objects in detail and
put toolbars into their own window. You can
insert intermediate points into connections more easily.

\subsection{Changes in Version 1.5}

\subsubsection*{Modifications}

The compatibility with Java 1.2 was improved.
Bugs in the simulation engine were fixed.
Some GUI problems were corrected.
The menu structure was cleaned up and simplified.

\subsubsection*{Additions}

A persistent database backing supports the deployment 
of the simulator in environment with high availability 
requirements.

Arcs can be B-splines. 
An alignment can be specified for every text figure.
Transitions and places can be refined.
Subnets can be coarsened. Nets can be merged.

Drawings are now autosaved. A backup copy of every file is kept.
Undo and redo commands were added. Search and replace commands
were added.

The architecture guide was added, which is a manual that
describes the most important internal algorithms
and data structures of Renew.


\subsection{Changes in Version 1.5.1}

\subsubsection*{Modifications}

This was a maintenance release that provided
bug fixes for the simulation engine.

\subsection{Changes in Version 1.5.2}

\subsubsection*{Modifications}

This was a maintenance release that provided
bug fixes for the install scripts and a performance improvement
of the simulation engine.


\subsection{Changes in Version 1.6}

\subsubsection*{Modifications}

Java 1.2 is now required for compiling and running Renew.
Bugs in the Java net parser, class loader and simulation engine
were fixed.
Shadow net system serialization and rendering has been fixed.
The windows menu is sorted alphabetically.
Windows can be de-iconified.

\subsubsection*{Additions}

A remote layer allows the separation of the user interface from the
simulation engine.
A net loader allows on-demand loading and compilation of nets
during a simulation.

A new transition inscription \texttt{manual} was introduced
for transitions that are not supposed to fire automatically
in ordinary running simulations.


\subsection{Changes in Version 2.0}

\subsubsection*{Removals}

The \texttt{RenewMode} interface provided by the GUI has been removed.
In consequence, the start scripts for the modes disappeared, too.
The channel, name and isolated node checks have been removed from the
\texttt{Net} menu because they need to be adopted to the new
simulator architecture.

\subsubsection*{Modifications}

Java 1.4 is now required for compiling and running Renew.
The application was decomposed into several plug-ins.
The simulation engine was restructured.
The GUI application classes were restructured and (partly) converted
to use the Swing package from the Java foundation classes.
The import and export menus have been restructured.
The handling of the various configuration properties has been
canonised by the plug-in system.
The class loader for custom classes has changed.

The \texttt{:net()} channel is no longer invoked implicitly on
instance creation.
The class \texttt{SequentialSimulator} is replaced by the
\texttt{NonConcurrentSimulator} without deadlock detection
feature.
The expanded token display feature has moved into the optional FS
plug-in, this option has no effect unless the FS plug-in is
installed.

\subsubsection*{Additions}

A plug-in system was added as bottom layer of the application.
The ability to switch simulator modes, net formalisms, the net loader
path, and the remote access feature on the fly was added.
A PNML-compatible export format was added.
The editor is able to load drawings from URLs.
The net loader can now search nets relative to the classpath.

\subsection{Changes in Version 2.0.1}

\subsubsection*{Modifications}

This was a maintenance release that provided
bug fixes for the install scripts 
and some redraw issues of the graphical net editor.

\subsubsection*{Additions}

An experimental \texttt{AppleUI} plug-in is available as optional
download. It provides rudimentary integration with the Mac OS look\&feel.

\subsection{Changes in Version 2.1}

\subsubsection*{Modifications}

Many error messages of the Java net compiler or about problems in a running
simulation became more detailed.
The command-line tool \texttt{ShadowTranslator} and the corresponding Ant
task now optionally include the specified formalism and a syntax check.
Fixed transition modes of bool net compiler.
Fixed manual transitions in saved simulation states.

The color and font attribute dialogues were improved.
Whitespace-only inscriptions are now deleted automatically (like empty
ones).
The GuiPrompt plug-in now provides a text area for command feedback.
The binding selection frame is now scrollable.
Tool windows and dialogues are now listed in the \texttt{Windows} menu.
Breakpoints pre-set via the \texttt{Net} menu are now visually tagged.
Fixed some drawing edit bugs in the GUI.
Fixed some rare deadlocks in the token game display.
Improved scrolling effect of mouse wheel in drawings.
Fixed handling of polygons.

Some important changes to configuration properties are documented in the
upgrade notes (see Section~\ref{sec:upgrade}).
Developers might also have a look there because of some code changes.

The set of Ant build files that come with the Renew source has undergone
some changes.
The build process now stores information about the compilation environment
with the plug-ins.
Distribution file names can now optionally include version information.
The \texttt{list} and \texttt{info} commands optionally display this
information.

\subsubsection*{Additions}

All Renew components (except the console prompt) now use the Apache Log4J
logging framework instead of Java console output. 
In the default configuration, informational and error messages are printed
to the console and logged to a file.
The simulation trace also goes to the logging framework (see
Section~\ref{subsec:log4jConfiguration}).
The new Logging plug-in provides a simulation trace window within the GUI.

The \texttt{Net step} option has been added to the simulation menu.

\subsection{Changes in Version 2.1.1}

\subsubsection*{Modifications}

This was a maintenance release that provides several minor bug fixes for
PNML export, \texttt{null} token display and access to public methods of
private (inner) classes in Java expressions.
In addition, this release is capable of reading drawing files created with
the later release 2.2 (with some minor exceptions).

\subsubsection*{Additions}

The AppleUI plugin now supports building a Mac OS X application bundle.

\subsection{Changes in Version 2.2}

\subsubsection*{Modifications}

Java 1.5 is now required for compiling and running Renew.

The Gui now uses the Graphics2D Framework that came with Java 1.2, so some of the figures might
be drawn a little bit different when it comes to size and style.
Tokens are now displayed on a white opaque background in the token game to
increase readability.
Scrolling now continues if the mouse is moved outside a drawing while a
button is pressed.

Modifier keys (Ctrl, Shift) have been added to several commands and tools
on drawing figures.
These enable users to resize figures to equal width and height, to adjust
polygon vertices at right angles with their adjacent edges, and to restrict
polygon transformations to either scaling or rotation.
Keyboard movement of figures with arrow keys can now be sped up using the
Shift modifier.

The behavior of Search and Replace has been fixed so that multiple instances of the search string 
in the same figure are now found and replaced correctly.
The Drawing Load Server has been restricted to accept local connection
requests only.

\subsubsection*{Additions}

Renew now includes and uses the VectorGraphics packages of the FreeHEP
project to accomplish graphical export of drawings. 
Additional supported export formats are PDF, SVG, and PNG.
EPS export now exports non-standard fonts correctly (at the price of larger
files).
EPS files now always have a rectangular white canvas.

A new pie figure allows to draw segments of arcs and ellipses.
Line styles (dotted, dashed, etc.) can now be applied to boxes, ellipses
and other figures with outlines.
A transparency attribute has been added to all figure, font and pen colors.
This breaks compatibility with older Renew versions, so that drawings saved with version 2.2 
can not be opened by previous versions (except release 2.1.1).
The transparency attribute is currently ignored when exporting drawings to EPS and this feature 
might not be implemented in future versions.
For drawings with transparency use one of the two new export formats SVG or
PDF, which handle transparency correctly.

The hotkey Ctrl+M now brings the menu and toolbar frame to front.
Added ``show net pattern/instance element'' options to the context menu in
the simulation trace window.
The net stub compiler now additionally supports stub objects that wrap
themselves around an existing net instance during instantiation (before, a
stub object always created its own net instance).

We provide a Mac OS X application bundle as well as configuration files for
the FreeDesktop (e.g. Gnome) environment that allow desktop integration
with separate icons and mime-types for Renew document files.
However, there is still no such support for the Windows family of operating
systems.

\subsubsection*{Relevant for developers only}

GUI and simulation have been separated so that they use different threads now. 
All calls to the simulation are decoupled and executed in specialized
simulation threads.
All calls to the GUI are delegated to or synchronized with the AWT event
thread.
Simulation threads can now be configured with a separate priority.
Loading of user-supplied classes in the context of simulations has been improved.

A new parameter ``netpath'' has been added to the Ant task to create shadow
net systems.
The Ant build environment has been enhanced to support separate source
code trees for JUnit tests and Cobertura coverage reports.
However, there still are nearly no test cases implemented.
Several tools that form the Renew build environment are now required in newer releases.
Please refer to the readme file in the source package.

\subsection{Changes in Version 2.3}
\label{sec:changes-2.3}

\subsubsection*{Modifications}

Java 1.6 is now required for compiling and running Renew.

Renew now includes and uses the 2.2 version of the FreeHEP project for
graphical exports of drawings.

Renew offers a better syntax check for Java reference net
models. If a Java inscription references a non-existing method or
field of an object, a proposal for existing methods or fields is
made instead of just pointing out the syntax error (see also
Section~\ref{sec:errors}).

Minor modifications to the graphical editor functionality of Renew
consist of the following. The names and colors of place figures are
now transferred to their virtual places. The editor prevents adding
more than one arc inscription by right-clicking on an arc with the
mouse as this happened rather by accident than on purpose. However, it
is still possible to add multiple arc inscriptions by using the
inscription tool.

\subsubsection*{Additions}

On startup Renew displays a splashscreen that gives information about
the loaded plugins.

There are two new entries in the File menu. The first addition is a
list of recently saved drawings. The second addition is the
possibility to open the \emph{Renew Navigator}, which allows to import
file folders and show their content in a tree view. A more detailed
description of how to use the navigator can be found in
Section~\ref{sec:navigator}.

It is now possible to define re-usable \emph{Net Components}. A net
component consists of a set of net elements that typically fulfill
some generic function and can be treated as a whole in a larger net
model. More details can be found in Section~\ref{sec:net-components}.

The background of expanded tokens in instance/simulation
drawings can be changed to be transparent by setting the property 
\texttt{de.\allowbreak{}renew.\allowbreak{}gui.\allowbreak{}no\allowbreak{}Token\allowbreak{}Background}.

Several keyboard shortcuts have been changed and more have been
added, especially for selecting the main drawing tools. A
comprehensive list of existing shortcuts can be found in
Appendix~\ref{sec:keyboard-shortcuts}.

\subsubsection*{Relevant for Developers only}

Generics are now used throughout the code.

The RMI functionality which was formerly included in the Simulator
plugin was extracted into a new Remote plugin.

The lock functionality was moved from the Simulator to the Util
plugin.

% Some bugs have been fixed in this version:
% \begin{itemize}
% \item Resolved deadlock in SimulatorPlugin facade (Bugzilla bug \#222).
% \item Fixed bug in handling boolean properties.
% \item Fixed handling plugin locations which had special characters in their filenames.
% \item Plugins can now add reference nets from multiple jar files
%   (previously only files from last jar file were loaded).
%   \TODO{mwe: verstehe ich nicht}
% \end{itemize}

There are several changes to the Ant build environment. The Ant target 
\texttt{clean} in the meta build file now iterates over
all subdirectories instead of having a fixed list of plugins. Source
files of nets (.rnw) can optionally be included in the generated
plugin archives (.jars) with the Ant target \texttt{rnw}. To activate
this function you need to set the property
\texttt{option.include.rnws} in your Ant properties (build.xml of the
plugin in question or ant local.properties). The property
\texttt{option.sns.compile} switches the syntax check for shadow net
files (.sns) on and off. The Ant target \texttt{javac} accepts an
encoding parameter which is set via the property
\texttt{option.compile.encoding} and defaults to utf-8. The Ant task
\texttt{createpng} allows to export net drawings to .png files.

%  Another change is related to the loading mechanism for plugins.
% \TODO{describe loading\\}

\subsection{Changes in Version 2.4}

\subsubsection*{Modifications}

The support for the \emph{.xrn} format is discontinued. We encourage the use
of PNML instead.  
We fixed the remote server connection (RMI) by providing
configuration (see the User Guide Section 2.6.)  
We have fixed the
simulation database backing and adapted the mechanism for MySQL with InnoDB.  
Remaining AWT dialogues have been converted to Swing.  
The Logging GUI has been improved by decoupling the Simulator from the GUI.  
The \emph{loadrenew} script now starts a regular Renew instance if the
connection to a Drawing Load Server is not possible.  
The desktop integration for Linux and Windows have been improved. We provide new unified icons for all operating systems.

Many minor bugs
have been fixed. Some of these were:
\begin{itemize}
\item Rare problems with locating nets relative to the \emph{classpath} have been
  solved.
\item Changes on the line style now affects all figures.
\item The font style \emph{underlined} now also affects small font sizes.
\item Export to PNML now always produces files in UTF-8 character
  encoding.
\item The Logging tab of the Configure Simulation dialog has been
  revised.
\item The log4j PatternLayout can now be set from within the GUI
\item Custom file appenders created in Logging GUI are now functional.
\item Net components are more robust if attached figures are manipulated.
\item It is now possible to escape whitespaces in command line commands. 
In that way it is possible to open drawings with whitespaces in the path 
from command line.
\end{itemize}

\subsubsection*{Additions}

The Navigator now offers a button to recursively expand a directory
sub-tree.  The Navigator now loads directories without locking the
GUI. PNML and ARM files are shown in the Navigator.  
The keyboard
shortcut \emph{Ctrl+Enter} closes the text editor overlay.  

The background transparency of EPS files can now be controlled by
setting the property \texttt{de.renew.io.export.eps-transparency}.
For Windows, the installation script \emph{installrenew.bat} creates reg
 files that associate and disassociate Renew drawing files with 
 \emph{loadrenew.bat} and register icons.  
 We provide deb packages for Debian-based systems.


\subsubsection*{Relevant for Developers}


We refactored large parts of the code base.  Many Java compiler
warnings have been resolved and Javadoc documentation have been
improved.  A few more JUnit tests have been introduced.  Logging and
the simulator have been partially reworked to allow deadlock-free
real-time (GUI) logging with only minimal time delay.  Several tools
that were originally mandatory to build Renew are now optional: these
are Cobertura, Jalopy, JUnit and Latex.

\subsection{Changes in Version 2.4.1}

This is a maintenance release that provides a fix for a race condition
that occurs - under rare conditions - during the termination of the
simulation.

% see https://tgiprojekte.informatik.uni-hamburg.de/issues/1527 for details

\subsection{Changes in Version 2.4.2}

This is a maintenance release that provides a fix for the import
of reference nets from PNML format (RefNet).

% see https://tgiprojekte.informatik.uni-hamburg.de/issues/3259 for detail

\subsection{Changes in Version 2.4.3}

This is a maintenance release that provides an update of the FreeHEP library
and a new version of the Mac OS X application bundle.
It fixes issues concerning the export functionality with newer Java versions.
This version requires at least Java 7.

% see https://tgiprojekte.informatik.uni-hamburg.de/issues/4327 for details

\subsection{Changes in Version 2.5}

\subsubsection*{Modifications}

We have modified several features of Renew.
Most obvious is the complete reimplementation of the Navigator plugin. 
It is now persistent, extensible and the tree view can now be filtered.
We optionally provide some convenient extensions, 
such as the integration of the drawing's diff feature (ImageNetDiff), 
which can now be triggered directly from the Navigator GUI.
The FreeHEP library has been upgraded to version 2.4.
The quick-draw\footnote{The possibility to quickly draw an arc and a n
  node by using the arc handle (see
  Section~\ref{subsubsec:toolTransition}).}  feature has been
improved, which results in a reduction to half the amount of mouse
clicks during quickly drawing net elements.
Some key-bindings are now configurable.

The log4j configuration and the configuration GUI have been improved. 
The individual log4j configuration file now resides in folder \emph{.renew}, 
which is located in the users home folder. The default location for log files moved there, as well.
The loading of plugins at startup can now be black-listed or white-listed.
The PDF export produces  PDF documents with bounding boxes; configuration has been fixed.
The grid can now be adapted and it can be activated as default.
Several console commands have been improved, including the
following plugin commands: \emph{list}, \emph{load},
\emph{unload} and also  export command \emph{ex}. Type \emph{help}
to print a synopsis of all commands.

\subsubsection*{Additions}
The Console plugin replaces the Prompt plugin. 
It employs the well-established JLine2 libray and provides several improvements:
tab-completion for commands and attributes, command history and editable command line.
The Quick Fix feature improves the reporting of syntax errors by
providing suitable proposals for remedies and their automatic
realization.
The Refactoring plugin (optional) provides features such as renaming
of variable or renaming of synchronous channels.
Drag \& drop now works for Renew drawings. Simply pull the drawing
over Renew's main window. Drawings and also folders can also be
dragged into the Navigator Window. Additionally you can add images to
a drawing by dragging it on the drawing's window.
Two new text tools have been added. One is the target text tool, which
allows to add hyperlinks to any drawing element. The targets can be
other model artifacts, for instance a net, or external resources
referenced by a URL. The hyperlink is activated by using the
\emph{Ctrl} modifier key together with a mouse click.
The comment tool allows to quickly add comments to a drawing element.
    
\subsubsection{Removals}
\begin{itemize}
\item Macao format has been removed, since its usability was very limited.
\item The PostScript export has been removed. Use EPS export or PDF export instead.
\end{itemize}



% Local Variables:
% mode: latex
% TeX-master:"renew.tex"
% End:



% Local Variables:
% mode: latex
% TeX-master:"renew.tex"
% End:


% LocalWords:  Runtime Solaris runtime MacOS JDK JHotDraw Petri API
% LocalWords:  classpath CLASSPATH startup instanceof VM netpath
% LocalWords:  formalisms failsafe boolean serializable stateful
% LocalWords:  backend Subdirectories prepending whitespace
